test
\documentclass[12pt]{article}
\usepackage[%
left=2.25cm,%
right=2.25cm,%
top=1.25cm,%
bottom=2.25cm,%
nohead%
]{geometry}

\usepackage{hyperref}
\usepackage{multicol}
\setcounter{secnumdepth}{4}
\setcounter{tocdepth}{4}
\usepackage{amsmath}
\usepackage{graphics}
\usepackage{subfig}
\linespread{1.4}
\usepackage{enumitem}% http://ctan.org/pkg/enumitem
\usepackage{tcolorbox}
\newtcolorbox{mybox}{boxrule=1pt,width=\textwidth,arc=0mm,colback=white}
\usepackage{xepersian}
\usepackage{shapepar}
\defpersianfont\nast{IranNastaliq}
\settextfont{B Nazanin}
\newtheorem{asl}{اصل}
%\renewcommand{\thethm}{\arabic{asl}}

%\title{}
\title{متن كامل قانون اساسی جمهوری اسلامی ایران}
\date{}
\begin{document}
	\maketitle
	\begin{center}
		\begin{large}
		بسم الله الرحمن الرحیم
		\\
		\vspace{3mm}
		لقد ارسلنا رسلنا بالبینات و انزلنا معهم الکتاب و المیزان لیقوم الناس بالقسط
		\end{large}
	\end{center}


\section{مقدمه}
‌قانون اساسی جمهوری اسلامی ایران مبین نهادهای فرهنگی‌، اجتماعی‌، سیاسی و اقتصادی جامعه ایران بر اساس اصول و ضوابط ‌اسلامی است که انعکاس خواست قلبی امت اسلامی می‌باشد. ماهیت انقلاب عظیم اسلامی ایران و روند مبارزه مردم مسلمان از ابتدا تا پیروزی که در شعارهای قاطع و کوبنده همه قشرهای مردم ‌تبلور می‌یافت این خواست اساسی را مشخص کرده و اکنون در طلیعه این پیروزی بزرگ‌، ملت ما با تمام وجود نیل به آن را می‌طلبد. ویژگی بنیادی این انقلاب نسبت به دیگر نهضت‌های ایران در سده، مکتبی و اسلامی بودن آن است‌. ملت مسلمان ایران پس از گذر از نهضت ضد استبدادی مشروطه و نهضت ضد استعماری ملی شدن نفت به این تجربه گرانبار دست یافت که علت اساسی و مشخص عدم موفقیت این نهضت‌ها، مکتبی نبودن مبارزات بوده است‌. گرچه در نهضت‌های اخیر خط فکری اسلامی و رهبری روحانیت مبارز سهم اصلی و اساسی را بر عهده داشت ولی به دلیل دور شدن این مبارزات ‌از مواضع اصیل اسلامی‌، جنبش‌ها به‌سرعت به رکود کشانده شد. از اینجا وجدان بیدار ملت به رهبری مرجع عالی‌قدر تقلید حضرت آیت‌الله‌العظمی امام خمینی ضرورت پیگیری خط نهضت اصیل‌ مکتبی و اسلامی را دریافت و این بار روحانیت مبارز کشور که همواره در صف مقدم نهضت‌های مردمی بوده و نویسندگان و روشنفکران متعهد با رهبری ایشان تحرک نوینی یافت‌. (آغاز نهضت اخیر ملت ایران در سال هزار و سیصد و هشتادودو هجری قمری‌ برابر با هزار و سیصد و چهل‌ویک هجری شمسی می‌باشد.)
\subsection*{طلیعه نهضت}

‌اعتراض درهم کوبنده امام خمینی به توطئه آمریکایی « انقلاب سفید » که گامی در جهت تثبیت پایه‌های حکومت استبداد و تحکیم ‌وابستگی‌های سیاسی‌، فرهنگی و اقتصادی ایران به امپریالیزم جهانی بود، عامل حرکت یکپارچه ملت گشت و متعاقب آن انقلاب عظیم و خون‌بار امت اسلامی در خردادماه 42 که در حقیقت نقطه آغاز شکوفایی این قیام شکوهمند و گسترده بود، مرکزیت امام را به‌عنوان رهبری اسلامی تثبیت و مستحکم نمود و علی‌رغم تبعید ایشان از ایران در پی اعتراض به قانون ننگین کاپیتولاسیون‌ (مصونیت مستشاران آمریکایی‌) پیوند مستحکم امت با امام همچنان استمرار یافت و ملت مسلمان و به‌ویژه روشنفکران متعهد و روحانیت مبارز راه خود را در میان تبعید و زندان‌، شکنجه و اعدام ادامه دادند. در این میان قشر آگاه و مسئول جامعه در سنگر مسجد، حوزه‌های علمیه و دانشگاه به روشنگری پرداخت و با الهام از مکتب انقلابی و پربار اسلام تلاش پیگیر و ثمربخشی را در بالا بردن سطح آگاهی و هوشیاری مبارزاتی و مکتبی ملت مسلمان آغاز کرد. رژیم استبداد که سرکوبی نهضت اسلامی را با حمله دژخیمانه به فیضیه و دانشگاه و همه کانون‌های پرخروش انقلاب آغاز نموده بود، به مذبوحانه‌ترین اقدامات ددمنشانه جهت رهایی از خشم انقلابی مردم‌، دست زد و در این میان‌، جوخه‌های اعدام‌، شکنجه‌های قرون‌وسطایی و زندان‌های درازمدت، بهایی بود که ملت مسلمان ما به نشانه عزم راسخ خود به ادامه مبارزه می‌پرداخت‌. خون صدها زن و مرد جوان و باایمان که سحرگاهان در میدان‌های تیر، فریاد «الله‌اکبر» سر می‌دادند یا در میان کوچه و بازار هدف گلوله‌های دشمن قرار می‌گرفتند، انقلاب اسلامی ‌ایران را تداوم بخشید، بیانیه‌ها و پیام‌های پی‌درپی امام به مناسبت‌های مختلف‌، آگاهی و عزم امت اسلامی را عمق و گسترش هر چه فزون‌تر داد.

\subsection*{حکومت اسلامی}
‌طرح حکومت اسلامی بر پایه ولایت‌فقیه که در اوج خفقان و اختناق رژیم استبدادی از سوی امام خمینی ارائه شد، انگیزه مشخص و منسجم نوینی را در مردم مسلمان ایجاد نمود و راه اصیل مبارزه مکتبی اسلام را گشود که تلاش مبارزان مسلمان و متعهد را در داخل و خارج کشور فشرده‌تر ساخت‌. در چنین خطی نهضت ادامه یافت تا سرانجام نارضایی‌ها و شدت خشم مردم براثر فشار و اختناق روزافزون در داخل و افشاگری انعکاس مبارزه به‌وسیله روحانیت و دانشجویان مبارز در سطح جهانی، بنیان‌های حاکمیت رژیم را بشدت متزلزل کرد و به‌ناچار رژیم و اربابانش مجبور به کاستن از فشار و اختناق و به‌اصطلاح‌ باز کردن فضای سیاسی کشور شدند تا به گمان خویش دریچه اطمینانی به‌منظور پیشگیری از سقوط حتمی خود بگشایند اما ملت برآشفته و آگاه و مصمم به رهبری قاطع و خلل‌ناپذیر امام‌، قیام پیروزمند و یکپارچه خود را به‌طور گسترده و سراسری آغاز نمود.

\subsection*{خشم ملت}

انتشار نامه توهین‌آمیزی به ساحت مقدس روحانیت و به‌ویژه‌ امام خمینی در 17 دی 1356 از طرف رژیم حاکم این حرکت را سریع‌تر نمود و باعث انفجار خشم مردم در سراسر کشور شد و رژیم برای مهار کردن آتش‌فشان خشم مردم کوشید این قیام معترضانه را با به خاک و خون کشیدن‌، خاموش کند اما این خود خون بیشتری در رگ‌های انقلاب جاری ساخت و تپش‌های پی‌درپی انقلاب در هفتم‌ها و چهلم‌های یادبود شهدای انقلاب‌، حیات و گرمی و جوشش یکپارچه و هر چه فزون‌تری به این نهضت در سراسر کشور بخشید و در ادامه و استمرار حرکت مردم تمامی سازمان‌های کشور با اعتصاب یکپارچه خود و شرکت در تظاهرات خیابانی در سقوط رژیم ‌استبدادی مشارکت فعالانه جستند، همبستگی گسترده مردان و زنان از همه اقشار و جناح‌های مذهبی و سیاسی در این مبارزه به طرز چشمگیری تعیین‌کننده بود و مخصوصاً زنان به شکل بارزی در تمامی صحنه‌های این جهاد بزرگ حضور فعال و گسترده‌ای داشتند، صحنه‌هایی از آن نوع که مادری را با کودکی در آغوش‌، شتابان به‌سوی میدان نبرد و لوله‌های مسلسل نشان می‌داد، بیانگر سهم عمده و تعیین‌کننده این قشر بزرگ جامعه در مبارزه بود.

\subsection*{بهایی که ملت پرداخت}
‌نهال انقلاب پس از یک سال و اندی مبارزه مستمر و پیگیر با باروری از خون بیش از شصت هزار شهید و صد هزار زخمی و معلول و با برجای نهادن میلیاردها تومان خسارت مالی در میان فریادهای: «استقلال‌، آزادی‌، حکومت اسلامی‌» به ثمر نشست و این نهضت عظیم که با تکیه‌بر ایمان و وحدت و قاطعیت رهبری در مراحل حساس و هیجان‌آمیز نهضت و نیز فداکاری ملت به پیروزی رسید موفق به درهم کوبیدن تمام محاسبات و مناسبات و نهادهای ‌امپریالیستی گردید که در نوع خود سرفصل جدیدی بر انقلاب‌های گسترده مردمی در جهان شد.21 و 22 بهمن سال یک هزار و سیصد و پنجاه‌وهفت روزهای فروریختن بنیاد شاهنشاهی شد و استبداد داخلی و سلطه خارجی ‌متکی بر آن را درهم شکست و با این پیروزی بزرگ طلیعه حکومت اسلامی که خواست دیرینه مردم مسلمان است نوید پیروزی نهایی را داد. ملت ایران به‌طور یکپارچه و با شرکت مراجع تقلید و علمای اسلام و مقام رهبری در همه‌پرسی جمهوری اسلامی تصمیم نهایی و قاطع خود را بر ایجاد نظام نوین جمهوری اسلامی اعلام کرد و با اکثریت 
\%98/2
به نظام جمهوری اسلامی رأی مثبت داد. اکنون قانون اساسی جمهوری اسلامی ایران به عنوان بیانگر نهادها و مناسبات سیاسی‌، اجتماعی‌، فرهنگی و اقتصادی جامعه باید راهگشای تحکیم پایه‌های حکومت اسلامی و ارائه‌دهنده طرح نوین نظام حکومتی بر ویرانه‌های نظام طاغوتی قبلی گردد.
 
 
\subsection*{شیوه حکومت در اسلام‌}
حکومت از دیدگاه اسلام‌، برخاسته از موضع طبقاتی و سلطه‌گری ‌فردی یا گروهی نیست بلکه تبلور آرمان سیاسی ملتی هم‌کیش و همفکر است که به خود سازمان می‌دهد تا در روند تحول فکری و عقیدتی راه خود را به‌سوی هدف نهایی (حرکت به‌سوی الله) بگشاید. ملت ما در جریان تکامل انقلابی خود از غبارها و زنگارهای طاغوتی زدوده شد و از آمیزه‌های فکری بیگانه خود را پاک نمود و به مواضع فکری و جهان‌بینی اصیل اسلامی بازگشت و اکنون بر آن است که با موازین اسلامی جامعه نمونه (اسوه‌) خود را بنا کند. بر چنین پایه‌ای‌، رسالت قانون اساسی این است که زمینه‌های اعتقادی نهضت را عینیت بخشد و شرایطی را به وجود آورد که در آن انسان با ارزش‌های والا و جهان‌شمول اسلامی پرورش ‌یابد. قانون اساسی با توجه به محتوای اسلامی انقلاب ایران که حرکتی برای پیروزی تمامی مستضعفین بر مستکبرین بود، زمینه تداوم این انقلاب را در داخل و خارج کشور فراهم می‌کند به‌ویژه در گسترش روابط بین‌المللی‌، با دیگر جنبش‌های اسلامی و مردمی می‌کوشد تا راه تشکیل امت واحد جهانی را هموار کند (ان هذه امتکم امه واحده و انا ربکم فاعبدون) و استمرار به مبارزه در نجات ملل محروم و تحت ستم در تمامی جهان قوام یابد. با توجه به ماهیت این نهضت بزرگ‌، قانون اساسی تضمین گر نفی هرگونه استبداد فکری و اجتماعی و انحصار اقتصادی می‌باشد و در خط گسستن از سیستم استبدادی‌ و سپردن سرنوشت مردم به دست خودشان تلاش می‌کند. (و یضع عنهم اصرهم و الاغلال التی کانت علیهم). در ایجاد نهادها و بنیادهای سیاسی که خود پایه تشکیل جامعه است ‌بر اساس تلقی مکتبی‌، صالحان عهده‌دار حکومت و اداره مملکت ‌می‌گردند (ان الارض یرثها عبادی الصالحون) و قانون‌گذاری که مبین ضابطه‌های مدیریت اجتماعی است بر مدار قرآن و سنت‌، جریان‌ می‌یابد؛ بنابراین نظارت دقیق و جدی از ناحیه اسلام‌شناسان عادل و پرهیزکار و متعهد (فقهای عادل‌) امری محترم و ضروری است و چون هدف از حکومت‌، رشد دادن انسان درحرکت به‌سوی ‌نظام الهی است (و الی الله مصیر) تا زمینه بروز و شکوفایی استعدادها به‌منظور تجلی ابعاد خدا گونگی انسان فراهم آید (تخلقوا باخلاق الله) و این جز در گرو مشارکت فعال و گسترده تمامی عناصر اجتماع در روند تحول جامعه نمی‌تواند باشد. با توجه به این جهت‌، قانون اساسی زمینه چنین مشارکتی را در تمام مراحل تصمیم‌گیری‌های سیاسی و سرنوشت‌ساز برای همه افراد اجتماع فراهم می‌سازد تا در مسیر تکامل انسان هر فردی خود دست‌اندرکار و مسئول رشد و ارتقاء و رهبری گردد که این همان تحقق حکومت مستضعفین در زمین خواهد بود. (و نرید ان نمن علی الذین استضعفوا فی الارض و نجعلهم ائمه و نجعلهم الوارثین)

\subsection*{ولایت فقیه}
عادل ‌بر اساس ولایت امر و امامت مستمر، قانون اساسی زمینه تحقق رهبری فقیه جامع‌الشرایطی را که از طرف مردم به عنوان رهبر شناخته می‌شود (مجاری الامور بید العلماء بالله الامناء علی حلاله و حرامه) آماده می‌کند تا ضامن عدم انحراف سازمان‌های مختلف از وظایف اصیل اسلامی خود باشد.

\subsection*{اقتصاد وسیله است نه هدف‌}
در تحکیم بنیادهای اقتصادی‌، اصل‌، رفع نیازهای انسان در جریان رشد و تکامل اوست نه همچون دیگر نظام‌های اقتصادی تمرکز و تکاثر ثروت و سودجویی‌، زیرا که در مکاتب مادی‌، اقتصاد خود هدف است و بدین‌جهت در مراحل رشد، اقتصاد عامل تخریب و فساد و تباهی می‌شود ولی در اسلام اقتصاد وسیله است و از وسیله ‌انتظاری جز کارایی ‌بهتر در راه وصول به هدف نمی‌توان داشت. با این دیدگاه برنامه اقتصاد اسلامی فراهم کردن زمینه مناسب برای بروز خلاقیت‌های متفاوت انسانی است و بدین‌جهت تأمین امکانات مساوی و متناسب و ایجاد کار برای همه افراد و رفع نیازهای ضروری جهت استمرار حرکت تکاملی او بر عهده حکومت اسلامی است.

\subsection*{زن در قانون اساسی‌}
در ایجاد بنیادهای اجتماعی اسلامی‌، نیروهای انسانی که تاکنون در خدمت استثمار همه‌جانبه خارجی بودند هویت اصلی و حقوق انسانی خود را بازمی‌یابند و در این بازیابی طبیعی است که زنان به دلیل ستم بیشتری که تاکنون از نظام طاغوتی متحمل شده‌اند استیفای حقوق آنان بیشتر خواهد بود. خانواده واحد بنیادین جامعه و کانون اصلی رشد و تعالی انسان است و توافق عقیدتی و آرمانی در تشکیل خانواده که زمینه‌ساز اصلی حرکت تکاملی و رشد یابنده انسان است اصل اساسی بوده و فراهم کردن امکانات جهت نیل به این مقصود از وظایف حکومت اسلامی است‌. زن در چنین برداشتی از واحد خانواده‌، از حالت‌(شی‌ء بودن‌) و یا (ابزار کار بودن) در خدمت اشاعه مصرف‌زدگی و استثمار، خارج شده و ضمن بازیافتن وظیفه خطیر و پرارج مادری در پرورش انسان‌های مکتبی پیشاهنگ و خود هم‌رزم مردان در میدان‌های فعال حیات می‌باشد و درنتیجه پذیرای مسئولیتی‌ خطیرتر و در دیدگاه اسلامی برخوردار از ارزش و کرامتی والاتر خواهد بود.
ارتش مکتبی ‌در تشکیل و تجهیز نیروهای دفاعی کشور توجه بر آن است که ایمان و مکتب‌، اساس و ضابطه باشد بدین‌جهت ارتش جمهوری اسلامی و سپاه پاسداران انقلاب در انطباق با هدف فوق شکل داده ‌می‌شوند و نه تنها حفظ و حراست از مرزها بلکه بار رسالت مکتبی یعنی جهاد در راه خدا و مبارزه در راه گسترش حاکمیت قانون خدا در جهان را نیز عهده‌دار خواهند بود. (و اعدوا لهم ما استطعتم من قوه و من رباط الخیل ترهبون به عدوالله و عدوکم و آخرین من دونهم).

\subsection*{قضا در قانون اساسی}
مسئله قضا در رابطه با پاسداری از حقوق مردم در خط حرکت اسلامی، به‌منظور پیشگیری از انحرافات موضعی در درون امت اسلامی امری است حیاتی‌، ازاین‌رو ایجاد سیستم قضائی بر پایه عدل اسلامی و متشکل از قضات عادل و آشنا به ضوابط دقیق دینی، پیش‌بینی شده است‌، این نظام به دلیل حساسیت بنیادی و دقت در مکتبی بودن آن لازم است به دور از هر نوع رابطه و مناسبات‌ ناسالم باشد. (و اذا حکمتم بین الناس ان تحکموا بالعدل)

\subsection*{قوه مجریه‌}
قوه مجریه به دلیل اهمیت ویژه‌ای که در رابطه با اجرای احکام و مقررات اسلامی به‌منظور رسیدن به روابط و مناسبات عادلانه حاکم بر جامعه دارد و همچنین ضرورتی که این مسئله حیاتی در زمینه‌سازی وصول به هدف نهایی حیات خواهد داشت، بایستی راهگشای ایجاد جامعه اسلامی باشد. نتیجتاً محصورشدن در هر نوع نظام دست و پاگیر پیچیده که وصول به این هدف را کند و یا خدشه‌دار کند از دیدگاه اسلامی نفی خواهد شد، بدین‌جهت نظام ‌بوروکراسی که زاییده و حاصل حاکمیت‌های طاغوتی است‌، بشدت طرد خواهد شد تا نظام اجرایی با کارایی بیشتر و سرعت افزون‌تر در اجرای تعهدات اداری به وجود آید.

\subsection*{وسایل ارتباط جمعی}
‌وسایل ارتباط‌جمعی (رادیو ـ تلویزیون‌) بایستی در جهت روند تکاملی انقلاب اسلامی در خدمت اشاعه فرهنگ اسلامی قرار گیرد و در این زمینه از برخورد سالم اندیشه‌های متفاوت بهره جوید و از اشاعه و ترویج خصلت‌های تخریبی و ضد اسلامی جدا پرهیز کند. پیروی از اصول چنین قانونی که آزادی و کرامت ابنای بشر را سرلوحه اهداف خود دانسته و راه رشد و تکامل انسان را می‌گشاید بر عهده همگان است و لازم است که امت مسلمان با انتخاب‌ مسئولین کاردان و مؤمن و نظارت مستمر بر کار آنان به‌طور فعالانه در ساختن جامعه اسلامی مشارکت جویند، به امید اینکه در بنای جامعه نمونه اسلامی (اسوه‌) که بتواند الگو و شهیدی بر همگی مردم جهان باشد موفق گردد. (و کذلک جعلناکم امه وسطا لتکونوا شهداء علی الناس)

\subsection*{نمایندگان‌}
مجلس خبرگان متشکل از نمایندگان مردم‌، کار تدوین قانون اساسی را بر اساس بررسی پیش‌نویس پیشنهادی دولت و کلیه پیشنهادهایی که از گروه‌های مختلف مردم رسیده بود در دوازده فصل که مشتمل بر یک‌صد و هفتادوپنج اصل می‌باشد در طلیعه پانزدهمین قرن هجرت پیغمبر اکرم صلی‌الله علیه واله و سلم بنیان‌گذار مکتب رهایی‌بخش اسلام با اهداف و انگیزه‌های مشروح فوق به پایان رساند، به این امید که این قرن‌، قرن حکومت‌ جهانی مستضعفین و شکست ‌تمامی مستکبرین گردد.


\section{اصول کلی}
\begin{asl}- 		
	حکومت ایران جمهوری اسلامی است که ملت ایران‌، بر اساس اعتقاد دیرینه‌اش به حکومت حق و عدل قرآن‌، در پی ‌انقلاب اسلامی پیروزمند خود به رهبری مرجع عالی‌قدر تقلید آیت‌الله‌العظمی امام خمینی، در همه‌پرسی دهم و یازدهم فروردین‌ماه یک هزار و سیصد و پنجاه‌وهشت هجری شمسی برابر با اول و دوم جمادی‌الاولی سال یک هزار و سیصد و نودونه هجری قمری با اکثریت   98/2 \% کلیه کسانی که حق رأی داشتند به آن رأی مثبت داد.
		
\end{asl}
\begin{asl}- 	
	جمهوری اسلامی‌، نظامی است برپایه ایمان به‌:
	\begin{enumerate}
		\item 
	خدای یکتا (لااله‌الاالله) و اختصاص حاکمیت و تشریع به او و لزوم تسلیم در برابر امر او.
		\item
	وحی الهی و نقش بنیادی آن در بیان قوانین‌.
		\item
	معاد و نقش سازنده آن در سیر تکاملی انسان به سوی خدا.
		\item
	عدل خدا در خلقت و تشریع‌.
	 	\item
	امامت و رهبری مستمر و نقش اساسی آن در تداوم انقلاب اسلام.
		\item
	کرامت و ارزش والای انسان و آزادی توأم با مسئولیت او در برابر خدا که از راه‌:
	\begin{enumerate}
		\item 
	اجتهاد مستمر فقهای جامع‌الشرایط بر اساس کتاب و سنت معصومین سلام‌الله علیهم اجمعین‌.	
		\item
	استفاده از علوم و فنون و تجارب پیشرفته بشری و تلاش در پیشبرد آن‌ها.
		\item
	نفی هرگونه ستمگری و ستم‌کشی و سلطه‌گری و سلطه‌پذیری‌، قسط و عدل و استقلال سیاسی و اقتصادی و اجتماعی و فرهنگی و همبستگی ملی را تأمین می‌کند.
	\end{enumerate}
	\end{enumerate}			
\end{asl}

\begin{asl}- 
	دولت جمهوری اسلامی ایران موظف است برای نیل به اهداف مذکور در اصل دوم‌، همه امکانات خود را برای امور زیر به کار برد:
		\begin{enumerate}	
		\item
	ایجاد محیط مساعد برای رشد فضایل اخلاقی بر اساس ایمان و تقوا و مبارزه با کلیه مظاهر فساد و تباهی‌.
		\item
	بالا بردن سطح آگاهی‌های عمومی در همه زمینه‌ها با استفاده‌ صحیح از مطبوعات و رسانه‌های گروهی و وسایل دیگر.
		\item
	آموزش‌وپرورش و تربیت‌بدنی رایگان برای همه‌، در تمام سطوح و تسهیل و تعمیم آموزش عالی‌.
		\item
	تقویت روح بررسی و تتبع و ابتکار در تمام زمینه‌های علمی‌، فنی‌، فرهنگی و اسلامی از طریق تأسیس مراکز تحقیق و تشویق ‌محققان‌.
		\item
	طرد کامل استعمار و جلوگیری از نفوذ اجانب‌.
		\item
	محو هرگونه استبداد و خودکامگی و انحصارطلبی‌.
		\item
	تأمین آزادی‌های سیاسی و اجتماعی در حدود قانون‌.
		\item
	مشارکت عامه مردم در تعیین سرنوشت سیاسی‌، اقتصادی‌، اجتماعی و فرهنگی خویش‌.
		\item
	رفع تبعیضات ناروا و ایجاد امکانات عادلانه برای همه‌، در تمام‌ زمینه‌های مادی و معنوی‌.
		\item
	ایجاد نظام اداری صحیح و حذف تشکیلات غیرضروری.
		\item
	تقویت کامل بنیه دفاع ملی از طریق آموزش نظامی عمومی برای حفظ استقلال و تمامیت ارضی و نظام اسلامی کشور.
		\item
	پی‌ریزی اقتصاد صحیح و عادلانه بر طبق ضوابط اسلامی‌ جهت ایجاد رفاه و رفع فقر و برطرف ساختن هر نوع محرومیت در زمینه‌های تغذیه و مسکن و کار و بهداشت و تعمیم بیمه‌.
		\item
	تأمین خودکفایی در علوم و فنون صنعت و کشاورزی و امور نظامی و مانند این‌ها.
		\item
	تأمین حقوق همه‌جانبه افراد از زن و مرد و ایجاد امنیت قضائی عادلانه برای همه و تساوی عموم در برابر قانون‌.
		\item
	توسعه و تحکیم برادری اسلامی و تعاون عمومی بین همه مردم.
		\item
	تنظیم سیاست خارجی کشور بر اساس معیارهای اسلام‌، تعهد برادرانه نسبت به همه مسلمانان و حمایت بی‌دریغ از مستضعفان‌ جهان‌.
		\end{enumerate}	
\end{asl}

\begin{asl}- 
کلیه قوانین و مقررات مدنی‌، جزائی‌، مالی‌، اقتصادی‌، اداری‌، فرهنگی‌، نظامی‌، سیاسی و غیر این‌ها باید بر اساس موازین اسلامی باشد. این اصل بر اطلاق یا عموم همه اصول قانون اساسی ‌و قوانین و مقررات دیگر حاکم است و تشخیص این امر بر عهده ‌فقهای شورای نگهبان است‌.
\end{asl}

\begin{asl}- 
در زمان غیبت حضرت ولی‌عصر «عجل‌الله تعالی فرجه‌» در جمهوری اسلامی ایران ولایت امر و امامت امت بر عهده فقیه عادل و باتقوا، آگاه به زمان‌، شجاع‌، مدیر و مدبر است که طبق اصل ‌یک‌صد و هفتم عهده‌دار آن می‌گردد.
\end{asl}

\begin{asl}- 
در جمهوری اسلامی ایران امور کشور باید به اتکا آرای عمومی اداره شود، از راه انتخابات‌: انتخاب رئیس‌جمهور، نمایندگان مجلس شورای اسلامی‌، اعضای شوراها و نظایر این‌ها، یا از راه همه‌پرسی در مواردی که در اصول دیگر این قانون معین ‌می‌گردد.
\end{asl}

\begin{asl}- 
طبق دستور قرآن کریم‌: «و امرهم شوری بینهم» و «شاورهم فی الامر» شوراها: مجلس شورای اسلامی‌، شورای استان‌، شهرستان‌، شهر، محل‌، بخش‌، روستا و نظایر این‌ها از ارکان تصمیم‌گیری و اداره امور کشورند.
موارد، طرز تشکیل و حدود اختیارات و وظایف شوراها را این ‌قانون و قوانین ناشی از آن معین می‌کند.
\end{asl}

\begin{asl}- 
در جمهوری اسلامی ایران دعوت به خیر، امربه‌معروف و نهی از منکر وظیفه‌ای است همگانی و متقابل بر عهده مردم نسبت به یکدیگر، دولت نسبت به مردم و مردم نسبت به دولت‌. شرایط و حدود و کیفیت آن را قانون معین می‌کند. «و المومنون و المومنات بعضهم اولیاء بعض یامرون بالمعروف و ینهون عن المنکر»
\end{asl}

\begin{asl}- 
در جمهوری اسلامی ایران آزادی و استقلال و وحدت و تمامیت ارضی کشور از یکدیگر تفکیک‌ناپذیرند و حفظ آن‌ها وظیفه دولت و آحاد ملت است‌. هیچ فرد یا گروه یا مقامی حق ندارد به نام استفاده از آزادی به استقلال سیاسی‌، فرهنگی‌، اقتصادی‌، نظامی و تمامیت ارضی ایران کمترین خدشه‌ای وارد کند و هیچ مقامی حق ندارد به نام حفظ استقلال و تمامیت ارضی کشور آزادی‌های مشروع را، هرچند با وضع قوانین و مقررات‌، سلب کند.
\end{asl}

\begin{asl}- 
ازآنجا که خانواده واحد بنیادی جامعه اسلامی است‌، همه قوانین و مقررات و برنامه‌ریزی‌های مربوط باید در جهت آسان کردن‌ تشکیل خانواده‌، پاسداری از قداست آن و استواری روابط خانوادگی ‌بر پایه حقوق و اخلاق اسلامی باشد.
\end{asl}

\begin{asl}- 
به حکم آیه کریمه «ان هذه امتکم امه واحده و انا ربکم فاعبدون» همه مسلمانان یک امت‌اند و دولت جمهوری اسلامی ایران موظف است سیاست کلی خود را بر پایه ائتلاف و اتحاد ملل اسلامی قرار دهد و کوشش پیگیر به عمل آورد تا وحدت سیاسی‌، اقتصادی و فرهنگی جهان اسلام را تحقق بخشد.
\end{asl}

\begin{asl}- 
دین رسمی ایران‌، اسلام و مذهب جعفری اثنی‌عشری است و این اصل الی‌الابد غیرقابل تغییراست و مذاهب دیگر اسلامی اعم از حنفی‌، شافعی‌، مالکی‌، حنبلی و زیدی دارای احترام کامل می‌باشند و پیروان این مذاهب در انجام مراسم مذهبی‌، طبق فقه خودشان آزادند و در تعلیم و تربیت دینی و احوال شخصیه (ازدواج‌، طلاق، ارث و وصیت‌) و دعاوی مربوط به آن در دادگاه‌ها رسمیت دارند و در هر منطقه‌ای که پیروان هر یک از این مذاهب اکثریت داشته باشند، مقررات محلی در حدود اختیارات شوراها بر طبق آن مذهب خواهد بود، با حفظ حقوق پیروان سایر مذاهب‌.
\end{asl}

\begin{asl}- 
ایرانیان زرتشتی‌، کلیمی و مسیحی تنها اقلیت‌های دینی شناخته می‌شوند که در حدود قانون در انجام مراسم دینی خود آزادند و در احوال شخصیه و تعلیمات دینی بر طبق آیین خود عمل می‌کنند.
\end{asl}

\begin{asl}- 
به حکم آیه شریه «لا ینهاکم الله عن الذین لم یقاتلوکم فی الدین و لم یخرجوکم من دیارکم ان تبروهم و تقسطوا الیهم ان الله یحب المقسطینَ» دولت جمهوری اسلامی ایران و مسلمانان موظف‌اند نسبت به افراد غیرمسلمان با اخلاق حسنه و قسط و عدل اسلامی عمل نمایند و حقوق انسانی آنان را رعایت کنند، این اصل در حق کسانی اعتبار دارد که بر ضد اسلام و جمهوری اسلامی ایران توطئه و اقدام نکنند.
\end{asl}

\section{زبان، خط، تاریخ و پرچم رسمی کشور}

\begin{asl}- 
زبان و خط رسمی و مشترک مردم ایران فارسی است‌. اسناد و مکاتبات و متون رسمی و کتب درسی باید با این زبان و خط باشد ولی استفاده از زبان‌های محلی و قومی در مطبوعات و رسانه‌های گروهی و تدریس ادبیات آن‌ها در مدارس‌، در کنار زبان فارسی آزاد است‌.
\end{asl}

\begin{asl}- 
ازآنجا که زبان قرآن و علوم و معارف اسلامی عربی است و ادبیات فارسی کاملاً با آن آمیخته است این زبان باید پس از دوره ابتدایی تا پایان دوره متوسطه در همه کلاس‌ها و در همه رشته‌ها تدریس شود.
\end{asl}

\begin{asl}- 
مبدأ تاریخ رسمی کشور، هجرت پیامبر اسلام (صلی‌الله علیه و آله وسلم‌) است و تاریخ هجری شمسی و هجری قمری هردو معتبر است اما مبنای کار ادارات دولتی هجری شمسی است‌. تعطیل رسمی هفتگی روز جمعه است‌.
\end{asl}

\begin{asl}- 
پرچم رسمی ایران به رنگ‌های سبز و سفید و سرخ با علامت مخصوص جمهوری اسلامی و شعار «الله‌اکبر» است‌.
\end{asl}

\section{حقوق ملت}
\begin{asl}- 
مردم ایران از هر قوم و قبیله که باشند از حقوق مساوی ‌برخوردارند و رنگ‌، نژاد، زبان و مانند این‌ها سبب امتیاز نخواهد بود.
\end{asl}

\begin{asl}- 
همه افراد ملت اعم از زن و مرد یکسان در حمایت قانون قرار دارند و از همه حقوق انسانی‌، سیاسی‌، اقتصادی‌، اجتماعی و فرهنگی با رعایت موازین اسلام برخوردارند.
\end{asl}

\begin{asl}- 
دولت موظف است حقوق زن را در تمام جهات با رعایت موازین اسلامی تضمین نماید و امور زیر را انجام دهد:
\begin{enumerate}
	\item 
	ایجاد زمینه‌های مساعد برای رشد شخصیت زن و احیای حقوق مادی و معنوی او.
	\item
	حمایت مادران‌، بالخصوص در دوران بارداری و حضانت فرزند، و حمایت از کودکان بی‌سرپرست‌.
	\item
	ایجاد دادگاه صالح برای حفظ کیان و بقای خانواده‌.
	\item
	ایجاد بیمه خاص بیوگان و زنان سالخورده و بی‌سرپرست‌.
	\item
	اعطای قیمومیت فرزندان به مادران شایسته در جهت غبطه آن‌ها در صورت نبودن ولی شرعی‌.
\end{enumerate}
\end{asl}

\begin{asl}- 
حیثیت‌، جان‌، مال‌، حقوق، مسکن و شغل اشخاص از تعرض مصون است مگر در مواردی که قانون تجویز کند.
\end{asl}

\begin{asl}- 
تفتیش عقاید ممنوع است و هیچ‌کس را نمی‌توان به‌صرف داشتن عقیده‌ای مورد تعرض و مؤاخذه قرار داد.
\end{asl}

\begin{asl}- 
نشریات و مطبوعات در بیان مطالب آزادند مگر آن که مخل به مبانی اسلام یا حقوق عمومی باشد. تفصیل آن را قانون معین می‌کند.
\end{asl}

\begin{asl}- 
بازرسی و نرساندن نامه‌ها، ضبط و فاش کردن مکالمات تلفنی‌، افشای مخابرات تلگرافی و تلکس‌، سانسور، عدم مخابره و نرساندن آن‌ها، استراق سمع و هرگونه تجسس ممنوع است مگر به حکم قانون‌.
\end{asl}

\begin{asl}- 
احزاب‌، جمعیت‌ها، انجمن‌های سیاسی و صنفی انجمن‌های اسلامی یا اقلیت‌های دینی شناخته‌شده آزادند، مشروط به این که اصول استقلال‌، آزادی‌، وحدت ملی‌، موازین اسلامی و اساس جمهوری اسلامی را نقض نکنند. هیچ‌کس را نمی‌توان از شرکت در آن‌ها منع کرد یا به شرکت در یکی از آن‌ها مجبور ساخت‌.
\end{asl}

\begin{asl}- 
تشکیل اجتماعات و راه‌پیمایی‌ها، بدون حمل سلاح، به شرط آن‌که مخل به مبانی اسلام نباشد آزاد است‌.
\end{asl}

\begin{asl}- 
هر کس حق دارد شغلی را که بدان مایل است و مخالف اسلام و مصالح عمومی و حقوق دیگران نیست برگزیند.
دولت موظف است با رعایت نیاز جامعه به مشاغل گوناگون برای همه افراد امکان اشتغال به کار و شرایط مساوی را برای احراز مشاغل ایجاد نماید.
\end{asl}

\begin{asl}- 
برخورداری از تأمین اجتماعی از نظر بازنشستگی‌، بیکاری‌، پیری‌، ازکارافتادگی، بی‌سرپرستی‌، در راه ماندگی، حوادث و سوانح و نیاز به خدمات بهداشتی و درمانی و مراقبت‌های پزشکی به‌صورت بیمه و غیره حقی است همگانی‌.
دولت مکلّف است طبق قوانین از محل درآمدهای عمومی و درآمدهای حاصل از مشارکت مردم‌، خدمات و حمایت‌های مالی فوق را برای یک‌یک افراد کشور تأمین کند.
\end{asl}

\begin{asl}- 
دولت موظف است وسایل آموزش و پروش رایگان را برای همه ملت تا پایان دوره متوسطه فراهم سازد و وسایل تحصیلات عالی را تا سرحد خودکفایی کشور به‌طور رایگان گسترش دهد.
\end{asl}

\begin{asl}- 
داشتن مسکن متناسب با نیاز، حق هر فرد و خانواده ایرانی است‌، دولت موظف است با رعایت اولویت برای آن‌ها که نیازمندترند بخصوص روستانشینان و کارگران زمینه اجرای این اصل را فراهم کند.
\end{asl}

\begin{asl}- 
هیچ‌کس را نمی‌توان دستگیر کرد مگر به حکم و ترتیبی که قانون معین می‌کند. در صورت بازداشت‌، موضوع اتهام باید با ذکر دلایل بلافاصله کتباً به متهم ابلاغ و تفهیم شود و حداکثر ظرف مدت بیست‌وچهار ساعت پرونده مقدماتی به مراجع صالحه قضائی ارسال و مقدمات محاکمه‌، در اسرع وقت فراهم گردد. متخلف از این اصل طبق قانون مجازات می‌شود.
\end{asl}

\begin{asl}- 
هیچ‌کس را نمی‌توان از محل اقامت خود تبعید کرد یا از اقامت در محل موردعلاقه‌اش ممنوع یا به اقامت در محلی مجبور ساخت‌، مگر در مواردی که قانون مقرر می‌دارد.
\end{asl}

\begin{asl}- 
دادخواهی حق مسلم هر فرد است و هر کس می‌تواند به منظور دادخواهی به دادگاه‌های صالح رجوع نماید، همه افراد ملت حق‌دارند این‌گونه دادگاه‌ها را در دسترس داشته باشند و هیچ‌کس را نمی‌توان از دادگاهی که به‌موجب قانون حق مراجعه به آن را دارد منع کرد.
\end{asl}

\begin{asl}- 
در همه دادگاه‌ها طرفین دعوا حق دارند برای خود وکیل انتخاب نمایند و اگر توانایی انتخاب وکیل را نداشته باشند باید برای آن‌ها امکانات تعیین وکیل فراهم گردد.
\end{asl}

\begin{asl}- 
حکم به مجازات و اجرای آن باید تنها از طریق دادگاه صالح و به موجب قانون باشد.
\end{asl}

\begin{asl}- 
اصل‌، برائت است و هیچ‌کس از نظر قانون مجرم شناخته نمی‌شود، مگر این که جرم او در دادگاه صالح ثابت گردد.
\end{asl}

\begin{asl}- 
هرگونه شکنجه برای گرفتن اقرار و یا کسب اطلاع ممنوع است‌، اجبار شخص به شهادت‌، اقرار یا سوگند مجاز نیست و چنین شهادت و اقرار و سوگندی فاقد ارزش و اعتبار است‌.
متخلف از این اصل طبق قانون مجازات می‌شود.
\end{asl}

\begin{asl}- 
هتک حرمت و حیثیت کسی که به حکم قانون دستگیر، بازداشت‌، زندانی یا تبعیدشده به هر صورت که باشد ممنوع و موجب مجازات است.
\end{asl}

\begin{asl}- 
هیچ‌کس نمی‌تواند اعمال حق خویش را وسیله اضرار به غیر یا تجاوز به منافع عمومی قرار دهد.
\end{asl}

\begin{asl}- 
تابعیت کشور ایران حق مسلم هر فرد ایرانی است و دولت نمی‌تواند از هیچ ایرانی سلب تابعیت کند، مگر به درخواست خود او یا در صورتی که به تابعیت کشور دیگری درآید.
\end{asl}

\begin{asl}- 
اتباع خارجه می‌توانند در حدود قوانین به تابعیت ایران درآیند و سلب تابعیت این‌گونه اشخاص در صورتی ممکن است که دولت دیگری تابعیت آن‌ها را بپذیرد یا خود آن‌ها درخواست کنند.
\end{asl}

\section{اقتصاد و امور مالی}
\begin{asl}- 
برای تأمین استقلال اقتصادی جامعه و ریشه‌کن کردن فقر و محرومیت و برآوردن نیازهای انسان در جریان رشد، با حفظ آزادگی او، اقتصاد جمهوری اسلامی ایران بر اساس ضوابط زیر استوار می‌شود:
\begin{enumerate}
	\item 
	تأمین نیازهای اساسی‌: مسکن‌، خوراک‌، پوشاک‌، بهداشت‌، درمان‌، آموزش‌وپرورش و امکانات لازم برای تشکیل خانواده برای همه‌.
	\item
	تأمین شرایط و امکانات کار برای همه به منظور رسیدن به اشتغال کامل و قرار دادن وسایل کار در اختیار همه کسانی که قادر به کارند ولی وسایل کار ندارند، در شکل تعاونی‌، از راه وام بدون بهره یا هر راه مشروع دیگر که نه به تمرکز و تداول ثروت در دست افراد و گروه‌های خاص منتهی شود و نه دولت را به صورت یک کارفرمای بزرگ مطلق درآورد. این اقدام باید با رعایت ضرورت‌های حاکم بر برنامه‌ریزی عمومی اقتصاد کشور در هر یک از مراحل رشد صورت گیرد.
	\item
	تنظیم برنامه اقتصادی کشور به صورتی که شکل و محتوا و ساعات کار چنان باشد که هر فرد علاوه بر تلاش شغلی‌، فرصت وتوان کافی برای خودسازی معنوی‌، سیاسی و اجتماعی و شرکت فعال در رهبری کشور و افزایش مهارت و ابتکار داشته باشد.	
	\item
	رعایت آزادی انتخاب شغل و عدم اجبار افراد به کاری معین و جلوگیری از بهره‌کشی از کار دیگری‌.	
	\item
	منع اضرار به غیر و انحصار و احتکار و ربا و دیگر معاملات باطل و حرام‌.
	\item
	منع اسراف و تبذیر در همه شئون مربوط به اقتصاد، اعم از مصرف، سرمایه‌گذاری‌، تولید، توزیع و خدمات‌.
	\item
	استفاده از علوم و فنون و تربیت افراد ماهر به نسبت احتیاج برای توسعه و پیشرفت اقتصاد کشور.
	\item
	جلوگیری از سلطه اقتصادی بیگانه بر اقتصاد کشور.
	\item
	تأکید بر افزایش تولیدات کشاورزی‌، دامی و صنعتی که نیازهای عمومی را تأمین کند و کشور را به مرحله خودکفایی برساند و از وابستگی برهاند.
\end{enumerate}
\end{asl}

\begin{asl}- 
نظام اقتصادی جمهوری اسلامی ایران بر پایه سه بخش دولتی‌، تعاونی و خصوصی با برنامه‌ریزی منظم و صحیح استوار است‌.
بخش دولتی شامل کلیه صنایع بزرگ‌، صنایع مادر، بازرگانی خارجی، معادن بزرگ‌، بانکداری‌، بیمه‌، تأمین نیرو، سدها و شبکه‌های بزرگ آب‌رسانی، رادیو و تلویزیون‌، پست و تلگراف و تلفن، هواپیمایی‌، کشتیرانی‌، راه و راه‌آهن و مانند این‌ها است که به‌صورت مالکیت عمومی و در اختیار دولت است‌.
بخش تعاونی شامل شرکت‌ها و مؤسسات تعاونی تولید و توزیع است که در شهر و روستا بر طبق ضوابط اسلامی تشکیل می‌شود.
بخش خصوصی شامل آن قسمت از کشاورزی‌، دامداری‌، صنعت‌، تجارت و خدمات می‌شود که مکمل فعالیت‌های اقتصادی دولتی و تعاونی است‌.
مالکیت در این سه بخش تا جایی که با اصول دیگر این فصل مطابق باشد و از محدوده قوانین اسلام خارج نشود و موجب رشد و توسعه اقتصادی کشور گردد و مایه زیان جامعه نشود موردحمایت قانون جمهوری اسلامی است‌.
تفصیل ضوابط و قلمرو و شرایط هر سه بخش را قانون معین می‌کند.
\end{asl}

\begin{asl}- 
انفال و ثروت‌های عمومی از قبیل زمین‌های موات یا رهاشده‌، معادن‌، دریاها، دریاچه‌ها، رودخانه‌ها و سایر آب‌های عمومی، کوه‌ها، دره‌ها، جنگل‌ها، نیزارها، بیشه‌های طبیعی‌، مراتعی که حریم نیست‌، ارث بدون وارث و اموال مجهول‌المالک و اموال عمومی که از غاصبین مسترد می‌شود، در اختیار حکومت اسلامی است تا بر طبق مصالح عامه نسبت به آن‌ها عمل نماید، تفصیل و ترتیب استفاده از هر یک را قانون معین می‌کند.
\end{asl}

\begin{asl}- 
هر کس مالک حاصل کسب‌وکار مشروع خویش است و هیچ‌کس نمی‌تواند به عنوان مالکیت نسبت به کسب‌وکار خود امکان کسب‌وکار را از دیگری سلب کند.
\end{asl}

\begin{asl}- 
مالکیت شخصی که از راه مشروع باشد محترم است. ضوابط آن را قانون معین می‌کند.
\end{asl}

\begin{asl}- 
در بهره‌برداری از منابع طبیعی و استفاده از درآمدهای ملی در سطح استان‌ها و توزیع فعالیت‌های اقتصادی میان استان‌ها و مناطق مختلف کشور، باید تبعیض در کار نباشد. به طوری که هر منطقه فراخور نیازها و استعداد رشد خود، سرمایه و امکانات لازم در دسترس داشته باشد.
\end{asl}

\begin{asl}- 
دولت موظف است ثروت‌های ناشی از ربا، غصب‌، رشوه‌، اختلاس‌، سرقت‌، قمار، سوءاستفاده از موقوفات‌، سوءاستفاده از مقاطعه‌کاری‌ها و معاملات دولتی‌، فروش زمین‌های موات و مباحات اصلی‌، دایر کردن اماکن فساد و سایر موارد غیر مشروع را گرفته و به صاحب حق رد کند و در صورت ‌معلوم نبودن او به بیت‌المال بدهد این حکم باید با رسیدگی و تحقیق و ثبوت شرعی به‌وسیله دولت اجرا شود.
\end{asl}

\begin{asl}- 
در جمهوری اسلامی‌، حفاظت محیط‌زیست که نسل امروز و نسل‌های بعد باید در آن حیات اجتماعی رو به رشدی داشته باشند، وظیفه عمومی تلقی می‌گردد. ازاین‌رو فعالیت‌های اقتصادی و غیر آن که با آلودگی محیط‌زیست یا تخریب غیرقابل‌جبران آن‌ ملازمه پیدا کند، ممنوع است‌.
\end{asl}

\begin{asl}- 
هیچ نوع مالیات وضع نمی‌شود مگر به موجب قانون. موارد معافیت و بخشودگی و تخفیف مالیاتی به موجب قانون مشخص می‌شود.
\end{asl}

\begin{asl}- 
بودجه سالانه کل کشور به ترتیبی که در قانون مقرر می‌شود از طرف دولت تهیه و برای رسیدگی و تصویب به مجلس شورای اسلامی تسلیم می‌گردد. هرگونه تغییر در ارقام بودجه نیز تابع مراتب مقرر در قانون خواهد بود.
\end{asl}

\begin{asl}- 
کلیه دریافت‌های دولت در حساب‌های خزانه‌داری کل متمرکز می‌شود و همه پرداخت‌ها در حدود اعتبارات مصوب به‌موجب قانون انجام می‌گیرد.
\end{asl}

\begin{asl}- 
دیوان محاسبات کشور مستقیماً زیر نظر مجلس شورای اسلامی می‌باشد. سازمان و اداره امور آن در تهران و مراکز استان‌ها به موجب قانون تعیین خواهد شد.
\end{asl}

\begin{asl}- 
دیوان محاسبات به کلیه حساب‌های وزارتخانه‌ها، مؤسسات‌، شرکت‌های دولتی و سایر دستگاه‌هایی که به نحوی از انحاء از بودجه کل کشور استفاده می‌کنند به ترتیبی که قانون مقرر می‌دارد رسیدگی یا حسابرسی می‌نماید که هیچ هزینه‌ای از اعتبارات مصوب تجاوز نکرده و هر وجهی در محل خود به مصرف رسیده باشد. دیوان محاسبات‌، حساب‌ها و اسناد و مدارک مربوطه را برابر قانون جمع‌آوری و گزارش تفریغ بودجه هرسال را به انضمام نظرات خود به مجلس شورای اسلامی تسلیم می‌نماید. این گزارش باید در دسترس عموم گذاشته شود.
\end{asl}

\section{حق حاکمیت ملّت و قوای ناشی از آن‌}
\begin{asl}- 
حاکمیت مطلق بر جهان و انسان از آن خداست هم او، انسان را بر سرنوشت اجتماعی خویش حاکم ساخته است. هیچ‌کس نمی‌تواند این حق الهی را از انسان سلب کند یا در خدمت ‌منافع فرد یا گروهی خاص قرار دهد و ملت این حق خداداد را از طرقی که در اصول بعد می‌آید اعمال می‌کند.
\end{asl}

\begin{asl}- 
قوای حاکم در جمهوری اسلامی ایران عبارت‌اند از: قوه مقننه‌، قوه مجریه و قوه قضائیه که زیر نظر ولایت مطلقه امر و امامت امت بر طبق اصول آینده این قانون اعمال می‌گردند. این قوا مستقل از یکدیگرند.
\end{asl}

\begin{asl}- 
اعمال قوه مقننه از طریق مجلس شورای اسلامی است که از نمایندگان منتخب مردم تشکیل می‌شود و مصوبات آن پس از طی مراحلی که در اصول بعد می‌آید برای اجرا به قوه مجریه و قضائیه ابلاغ می‌گردد.
\end{asl}

\begin{asl}- 
در مسائل بسیار مهم اقتصادی‌، سیاسی‌، اجتماعی ‌و فرهنگی ممکن است اعمال قوه مقننه از راه همه‌پرسی و مراجعه مستقیم به آرای مردم صورت گیرد. درخواست مراجعه به آرای عمومی باید به تصویب دوسوم مجموع نمایندگان مجلس برسد.
\end{asl}

\begin{asl}- 
اعمال قوه مجریه جز در اموری که در این قانون مستقیماً بر عهده رهبری گذارده شده‌، از طریق رئیس‌جمهور و وزراست.
\end{asl}

\begin{asl}- 
اعمال قوه قضائیه به‌وسیله دادگاه‌های دادگستری است که باید طبق موازین اسلامی تشکیل شود و به حل و فصل دعاوی و حفظ حقوق عمومی و گسترش و اجرای عدالت و اقامه حدود الهی بپردازد.
\end{asl}

\begin{asl}- 
مجلس شورای اسلامی از نمایندگان ملت که به طور مستقیم و با رأی مخفی انتخاب می‌شوند، تشکیل می‌گردد. شرایط انتخاب‏کنندگان و انتخاب شوندگان و کیفیت انتخابات را قانون معین خواهد کرد.
\end{asl}

\begin{asl}- 
دوره نمایندگی مجلس شورای اسلامی چهار سال است. انتخابات هر دوره باید پیش از پایان دوره قبل برگزار شود به طوری که کشور در هیچ زمان بدون مجلس نباشد.
\end{asl}

\begin{asl}- 
عده نمایندگان مجلس شورای اسلامی دویست و هفتاد نفر است و از تاریخ همه‌‏پرسی سال یک هزار و سیصد و شصت‌وهشت هجری شمسی پس از هر ده سال، با در نظر گرفتن عوامل انسانی، سیاسی، جغرافیایی و نظایر آن‌ها حداکثر بیست نفر نماینده می‌تواند اضافه شود. زرتشتیان و کلیمیان هرکدام یک نماینده و مسیحیان آشوری و کلدانی مجموعاً یک نمایند و مسیحیان ارمنی جنوب و شمال هرکدام یک نماینده انتخاب می‌کنند. محدوده حوزه‏های انتخابیه و تعداد نمایندگان را قانون معین می‌کند.
\end{asl}

\begin{asl}- 
پس از برگزاری انتخابات، جلسات مجلس شورای اسلامی با حضور دوسوم مجموع نمایندگان رسمیت می‌یابد و تصویب طرح‌ها و لوایح طبق آیین‏نامه مصوب داخلی انجام می‌گیرد مگر در مواردی که در قانون اساسی نصاب خاصی تعیین شده باشد. برای تصویب آیین‏نامه داخلی موافقت دوسوم حاضران لازم است.
\end{asl}

\begin{asl}- 
ترتیب انتخاب رئیس و هیئت‌رئیسه مجلس و تعداد کمیسیون‌ها و دوره تصدی آن‌ها و امور مربوط به مذاکرات و انتظامات مجلس به‌وسیله آیین‏نامه داخلی مجلس معین می‌گردد.
\end{asl}

\begin{asl}- 
سوگند نمایندگان مجلس
\\
نمایندگان باید در نخستین جلسه مجلس به ترتیب زیر سوگند یاد کنند و متن قسم‏نامه را امضاء نمایند. 
\\
بسم الله الرحمن الرحیم
\\
«من در برابر قرآن مجید، به خدای قادر متعال سوگند یاد می‌کنم و با تکیه بر شرف انسانی خویش تعهد می‌نمایم که پاسدار حریم اسلام و نگاهبان دستاوردهای انقلاب اسلامی ملت ایران و مبانی جمهوری اسلامی باشم، ودیعه‏ای را که ملت به ما سپرده به عنوا ن امینی عادل پاسداری کنم و در انجام وظایف وکالت، امانت و تقوی را رعایت نمایم و همواره به استقلال و اعتلای کشور و حفظ حقوق ملت و خدمت به مردم پایبند باشم، از قانون اساسی دفاع کنم و در گفته‏ها و نوشته‏ها و اظهارنظرها، استقلال کشور و آزادی مردم و تأمین مصالح آن‌ها را مدنظر داشته باشم.»
\\
نمایندگان اقلیت‌های دینی این سوگند را با ذکر کتاب آسمانی خود یاد خواهند کرد.

\end{asl}

\begin{asl}- 
در زمان جنگ و اشغال نظامی کشور به پیشنهاد رئیس‌جمهور و تصویب سه‌چهارم مجموع نمایندگان و تأیید شورای نگهبان از انتخابات نقاط اشغال‌شده یا تمامی مملکت برای مدت معینی متوقف می‌شود و در صورت عدم تشکیل مجلس جدید، مجلس سابق همچنان به کار خود ادامه خواهد داد. 
\end{asl}

\begin{asl}- 
انتشار مذاکرات مجلس از رادیو و روزنامه رسمی 
\\
مذاکرات مجلس شورای ملی باید علنی باشد و گزارش کامل آن از طریق رادیو و روزنامه رسمی برای اطلاع عموم منتشر شود. در شرایط اضطراری، در صورتی که رعایت امنیت کشور ایجاب کند، به تقاضای رئیس‌جمهور یا یکی از وزرا یا ده نفر از نمایندگان، جلسه غیرعلنی تشکیل می‏شود. مصوبات جلسه غیرعلنی در صورتی معتبر است که با حضور شورای نگهبان به تصویب سه‌چهارم مجموع نمایندگان برسد. گزارش و مصوبات این جلسات باید پس از برطرف شدن شرایط اضطراری برای اطلاع عموم منتشر گردد. 

\end{asl}

\begin{asl}- 
رئیس‌جمهور و معاونان او و وزیران به اجتماع یا با انفراد حق شرکت در جلسات علنی مجلس را دارند و می‌توانند مشاوران خود را همراه داشته باشند و در صورتی که نمایندگان لازم بدانند، وزرا مکلف به حضورند و هرگاه تقاضا کنند مطالبشان استماع می‌شود. 
\end{asl}

\begin{asl}- 
مجلس شورای اسلامی در عموم مسائل د ر حدود مقرر در قانون اساسی می‌تواند قانون وضع کند. 
\end{asl}

\begin{asl}- 
مجلس شورای اسلامی نمی‌تواند قوانینی وضع کند که با اصول و احکام مذهب رسمی کشور یا قانون اساسی مغایرت داشته باشد. تشخیص این امر به ترتیبی که در اصل نود و ششم آمده بر عهده شورای نگهبان است.  
\end{asl}

\begin{asl}- 
شرح و تفسیر قوانین عادی در صلاحیت مجلس شورای اسلامی است. مفاد این اصل مانع از تفسیری که دادستان، در مقام تمیز حق، از قوانین می‌کنند نیست. 
\end{asl}

\begin{asl}- 
لوایح قانونی پس از تصویب هیئت‌وزیران به مجلس تقدیم می‌شود و طرح‌های قانونی به پیشنهاد حداقل پانزده نفر از نمایندگان، در مجلس شورای اسلامی قابل‌طرح است. 
\end{asl}

\begin{asl}- 
طرح‏های قانونی و پیشنهادها و اصلاحاتی که نمایندگان در خصوص لوایح قانونی عنوان می‌کنند و به تقلیل درآمد عمومی یا افزایش هزینه عمومی می‌انجامد، در صورتی قابل‌طرح در مجلس است که در آن طریق جبران کاهش درآمد یا تأمین هزینه جدید نیز معلوم شده باشد. 
\end{asl}

\begin{asl}- 
مجلس شورای اسلامی حق تحقیق و تفحص در تمام امور کشور را دارد. 
\end{asl}

\begin{asl}- 
عهدنامه‏ها، مقاوله نامه‌ها، قراردادها و موافقت‏نامه‏‏های بین‏المللی باید به تصویب مجلس شورای اسلامی برسد. 
\end{asl}

\begin{asl}- 
هرگونه تغییر در خطوط مرزی ممنوع است مگر اصلاحات جزئی با رعایت مصالح کشور، به شرط این که یک‌طرفه نباشد و به استقلال و تمامیت ارضی کشور لطمه نزدن و به تصویب چهارپنجم مجموع نمایندگان مجلس شورای اسلامی برسد. 
\end{asl}

\begin{asl}- 
برقراری حکومت‌نظامی ممنوع است. در حالت جنگ و شرایط اضطراری نظیر آن، دولت حق دارد با تصویب مجلس شورای اسلامی موقتاً محدودیت‌های ضروری را برقرار نماید، ولی مدت آن به‌هرحال نمی‌تواند بیش از سی روز باشد و در صورتی که ضرورت همینان باقی باشد دولت موظف است مجدداً از مجلس کسب مجوز کند. 
\end{asl}

\begin{asl}- 
گرفتن و دادن وام یا کمک‌های بدون عوض داخلی و خارجی از طرف دولت باید با تصویب مجلس شورای اسلامی باشد. 
\end{asl}

\begin{asl}- 
دادن امتیاز تشکیل شرکت‌ها و مؤسسات در امور تجارتی و صنعتی و کشاورزی و معادن و خدمات به خارجیان مطلقاً ممنوع است. 
\end{asl}

\begin{asl}- 
استخدام کارشناسان خارجی از طرف دولت ممنوع است مگر در موارد ضرورت با تصویب مجلس شورای اسلامی. 
\end{asl}

\begin{asl}- 
بناها و اموالی دولتی که از نفایش ملی باشد قابل‌انتقال به غیر نیست مگر با تصویب مجلس شورای اسلامی آن‌هم در صورتی که از نفایش منحصربه‌فرد نباشد. 
\end{asl}

\begin{asl}- 
هر نماینده در برابر تمام ملت مسئول است و حق دارد در همه مسائل داخلی و خارجی کشور اظهارنظر نماید.  
\end{asl}

\begin{asl}- 
سمت نمایندگی قائم به شخص است و قابل‌واگذاری به دیگری نیست. مجلس نمی‌تواند اختیار قانون‌گذاری را به شخص یا هیئتی واگذار کند ولی در موارد ضروری می‌تواند اختیار وضع بعضی از قوانین را با رعایت اصل هفتاد و دوم به کمیسیون‌های داخلی خود تفویض کند، در این صورت این قوانین در مدتی که مجلس تعیین می‌نماید به صورت آزمایشی اجرا می‌شود و تصویب نهایی آن‌ها با مجلس خواهد بود. همچنین مجلس شورای اسلامی می‌تواند تصویب دائمی اساسنامه سازمان‌ها، شرکت‌ها، مؤسسات دولتی یا وابسته به دولت را با رعایت اصل هفتاد و دوم به کمیسیون‌های ذی‌ربط واگذار کند و یا اجازه تصویب آن‌ها را به دولت بدهد. در این صورت مصوبات دولت نباید با اصول و احکام مذهب رسمی کشور و یا قانون اساسی مغایرت داشته باشد، تشخیص این امر به ترتیب مذکور در اصل نود و ششم با شورای نگهبان است. علاوه بر این، مصوبات دولت نباید مخالفت قوانین و مقررات عمومی کشور باشد و به منظور بررسی و اعلام عدم مغایرت آن‌ها با قوانین مزبور باید ضمن ابلاغ برای اجرا به اطلاع رئیس مجلس شورای اسلامی برسد. 
\end{asl}

\begin{asl}- 
نمایندگان مجلس در مقام ایفای وظایف نمایندگی در اظهارنظر و رأی خود کاملاً آزادند و نمی‌توان آن‌ها را به سبب نظراتی که در مجلس اظهار کرده‏اند یا آرایی که در مقام ایفای وظایف نمایندگی خود داده‏اند تعقیب یا توقیف کرد.‏ 
\end{asl}

\begin{asl}- 
رئیس‌جمهور برای هیئت‌وزیران پس از تشکیل و پیش از هر اقدام دیگر باید از مجلس رأی اعتماد بگیرد. در دوران تصدی نیز در مورد مسائل مهم و مورد اختلاف می‌تواند از مجلس برای هیئت‌وزیران تقاضای رأی اعتماد کند. 
\end{asl}

\begin{asl}- 
در هر مورد که حداقل یک‌چهارم کل نمایندگان مجلس شورای اسلامی از رئیس‌جمهور و یا هر یک از نمایندگان از وزیر مسئول، درباره یکی از وظایف آنان سؤال کنند، رئیس‌جمهور یا وزیر موظف است در مجلس حاضر شود و به سؤال جواب دهد و این جواب نباید در مورد رئیس‌جمهور بیش از یک ماه و در مورد وزیر بیش از ده روز به تأخیر افتاد مگر با عذر موجه به تشخیص مجلس شورای اسلامی. 
\end{asl}

\begin{asl}- 
استیضاح وزیران و رئیس‌جمهور 
\\
نمایندگان مجلس شورای اسلامی می‌توانند در مواردی که لازم می‌دانند هیئت‌وزیران یا هر یک از وزرا را استیضاح کنند، استیضاح وقتی قابل‌طرح در مجلس است که با امضای حداقل ده نفر از نمایندگان به مجلس تقدیم شود. هیئت‌وزیران یا وزیر مورد استیضاح باید ظرف مدت ده روز پس از طرح آن در مجلس حاضر شد و به آن پاسخ گوید و از مجلس رأی اعتماد بخواهد. در صورت عدم حضور هیئت‌وزیران یا وزیر برای پاسخ، نمایندگان مزبور درباره استیضاح خود توضیحات لازم را می‌دهند و در صورتی که مجلس مقتضی بداند اعلام رأی عدم اعتماد خواهد کرد. اگر مجلس رأی اعتماد نداد هیئت‌وزیران یا وزیران یا وزیر مورد استیضاح عزل می‌شود. در هر دو صورت وزرای مورد استیضاح نمی‌توانند در هیئت‌وزیرانی که بلافاصله بعد از آن تشکیل می‌شود عضویت پیدا کنند. 
\\
در صورتی که حداقل یک‌سوم از نمایندگان مجلس شورای اسلامی رئیس‌جمهور را در مقام اجرای وظایف مدیریت قوه مجریه و اداره امور اجرایی کشور مورد استیضاح قرار دهند، رئیس‌جمهور باید ظرف مدت یک ماه پس از طرح آن در مجلس حاضر شود و در خصوص مسائل مطرح‌شده توضیحات کافی بدهد. در صورتی که پس از بیانات نمایندگان مخالف و موافق و پاسخ رئیس‌جمهور، اکثریت دوسوم کل نمایندگان به عدم کفایت رئیس‌جمهور رأی دادند مراتب جهت اجرای بند ده اصل یک‌صد و دهم به اطلاع مقام رهبری می‌رسد. 

\end{asl}

\begin{asl}- 
هر کس شکایتی از طرز کار مجلس یا قوه مجریه یا قوه قضاییه داشته باشد، می‌تواند شکایت خود را کتباً به مجلس شورای اسلامی عرضه کند. مجلس موظف است به این شکایات رسیدگی کند و پاسخ کافی دهد و در مواردی که شکایت به قوه مجریه و یا قوه قضاییه مربوط است رسیدگی و پاسخ کافی از آن‌ها بخواهد و در مدت متناسب نتیجه را اعلام نماید و در موردی که مربوط به عموم باشد به اطلاع عامه برساند.‏
\end{asl}

\begin{asl}- 
شورای نگهبان 
\\
به منظور پاسداری از احکام اسلام و قانون اساسی از نظر عدم مغایرت مصوبات، مجلس شورای اسلامی با آن‌ها، شورایی به نام شورای نگهبان با ترکیب زیر تشکیل می‌شود:
\renewcommand{\labelitemi}{$-$}
\begin{itemize}
	\item 
شش نفر از فقهای عادل و آگاه به مقتضیات زمان و مسائل روز. انتخاب این عده با مقام رهبری است. 
	\item
	شش نفر حقوقدان، در رشته‏های مختلف حقوقی، از میان حقوقدانان مسلمانی که به‌وسیله رئیس قوه قضاییه به مجلس شورای اسلامی معرف می‌شوند و با رأی مجلس انتخاب می‌گردند.
\end{itemize}
\end{asl}

\begin{asl}- 
اعضای شورای نگهبان برای مدت شش سال انتخاب می‏‌شوند ولی در نخستین دوره پس از گذشتن سه سال، نیمی از اعضای هر گروه به قید قرعه تغییر می‌یابند و اعضای تازه‏ای به جای آن‌ها انتخاب می‌شوند.  
\end{asl}

\begin{asl}- 
مجلس شورای اسلامی بدون وجود شورای نگهبان اعتبار قانونی ندارد مگر در مورد تصویب اعتبارنامه نمایندگان و انتخاب شش نفر حقوقدان اعضای شورای نگهبان. 
\end{asl}

\begin{asl}- 
کلیه مصوبات مجلس شورای اسلامی باید به شورای نگهبان فرستاده شود. شورای نگهبان موظف است آن را حداکثر ظرف ده روز از تاریخ وصول از نظر انطباق بر موازین اسلام و قانون اساسی موردبررسی قرار دهد و ینانیه آن را مغایر ببیند برای تجدیدنظر به مجلس بازگرداند. در غیر این صورت مصوبه قابل‌اجرا است. 
\end{asl}

\begin{asl}- 
در مواردی که شورای نگهبان مدت ده روز را برای رسیدگی و اظهارنظر نهایی کافی نداند، می‌تواند از مجلس شورای اسلامی حداکثر برای ده روز دیگر با ذکر دلیل خواستار تمدید وقت شود. 
\end{asl}

\begin{asl}- 
تشخیص عدم مغایرت مصوبات مجلس شورای اسلامی با احکام اسلام با اکثریت فقهای شورای نگهبان و تشخیص عدم تعارض آن‌ها با قانون اساسی بر عهده اکثریت همه اعضای شورای نگهبان است. 
\end{asl}

\begin{asl}- 
اعضای شورای نگهبان به منظور تسریع در کار می‌توانند هنگام مذاکره درباره لایحه یا طرح قانونی در مجلس حاضر شوند و مذاکرات را استماع کنند؛ اما وقتی طرح یا لایحه‏ای فوری در دستور کار مجلس قرار گیرد، اعضای شورای نگهبان باید در مجلس حاضر شوند و نظر خود را اظهار نمایند. 
\end{asl}

\begin{asl}- 
تفسیر قانون اساسی به عهده شورای نگهبان است که با تصویب سه‌چهارم آنان انجام می‌شود. 
\end{asl}

\begin{asl}- 
نظارت شورای نگهبان 
\\
شورای نگهبان نظارت بر انتخابات مجلس خبرگان رهبری، ریاست جمهوری، مجلس شورای اسلامی و مراجعه به آراء عمومی و همه‌‏پرسی را بر عهده دارد. 
\end{asl}

\begin{asl}- 
برای پیشبرد سریع برنامه‏‌های اجتماعی، اقتصادی، عمرانی، بهداشتی، فرهنگی، آموزشی و سایر امور رفاهی از طریق همکاری مردم با توجه به مقتضیات محلی، اداره امور هر روستا، بخش، شهر، شهرستان یا استان با نظارت شورایی به نام شورای ده، بخش، شهر، شهرستان یا استان صورت می‌گیرد که اعضای آن را مردم همان محل انتخاب می‌کنند. شرایط انتخاب‏‌کنندگان و انتخاب شوندگان و حدود وظایف و اختیارات و نحوه انتخاب و نظارت شوراهای مذکور و سلسله‌مراتب آن‌ها را که باید با رعایت اصول وحدت ملی و تمامیت ارضی و نظام جمهوری اسلامی و تابعیت حکومت مرکزی باشد قانون معین می‌کند.  
\end{asl}

\begin{asl}- 
به منظور جلوگیری از تبعیض و جلب همکاری در تهیه برنامه‏‌های عمرانی و رفاهی استان‌ها و نظارت بر اجرای هماهنگ آن‌ها، شورای عالی استان‌ها مرکب از نمایندگان شوراهای استان‌ها تشکیل می‌شود. نحوه تشکیل و وظایف این شورا را قانون معین می‌کند. 
\end{asl}

\begin{asl}- 
شورای عالی استان‌ها حق دارد در حدود وظایف خود طرح‌هایی تهیه و مستقیماً یا از طریق دولت به مجلس شورای اسلامی پیشنهاد کند. این طرح‌ها باید در مجلس موردبررسی قرار گیرد. 
\end{asl}

\begin{asl}- 
استانداران، فرمانداران، بخشداران و سایر مقامات کشوری که از طرف دولت تعیین می‌شوند در حدود اختیارات شوراها ملزم به رعایت تصمیمات آن‌ها هستند. 
\end{asl}

\begin{asl}- 
به منظور تأمین قسط اسلامی و همکاری در تهیه برنامه‌‏ها و ایجاد هماهنگی در پیشرفت امور در واحدهای تولیدی، صنعتی و کشاورزی، شوراهایی مرکب از نمایندگان کارگران و دهقانان و دیگر کارکنان و مدیران و در واحدهای آموزشی، اداری، خدماتی و مانند این‌ها شوراهایی مرکب از نمایندگان اعضاء این واحدها تشکیل می‌شود. چگونگی تشکیل این شوراها و حدود وظایف و اختیارات آن‌ها را قانون معین می‌کند. 
\end{asl}

\begin{asl}- 
تصمیمات شوراها نباید مخالف موازین اسلام و قوانین کشور باشد. 
\end{asl}

\begin{asl}- 
انحلال شوراها جز در صورت انحراف از وظایف قانونی ممکن نیست. مرجع تشخیص انحراف و ترتیب انحلال شوراها و طرز تشکیل مجدد آن‌ها را قانون معین می‌کند. شورا در صورت اعتراض به انحلال حق دارد به دادگاه صالح شکایت کند و دادگاه موظف است خارج از نوبت به آن رسیدگی کند. 
\end{asl}

\begin{asl}- 
خبرگان رهبری 
\\
پس از مرجع عالی‌قدر تقلید و رهبر کبیر انقلاب جهانی اسلام و بنیان‌گذار جمهوری اسلامی ایران حضرت آیت‏الله‏‌العظمی امام خمینی «قدس سره‏‌الشریف» که از طرف اکثریت قاطع مردم به مرجعیت و رهبری شناخته و پذیرفته شدند، تعیین رهبر به عهده خبرگان منتخب مردم است. خبرگان رهبری درباره همه فقهای واجد شرایط مذکور در اصول پنجم و یک‌صد و نهم بررسی و مشورت می‌کنند. هرگاه یکی از آنان را اعلم به احکام و موضوعات فقهی یا مسائل سیاسی و اجتماعی یا دارای مقبولیت عامه یا واجد برجستگی خاص در یکی از صفات مذکور در اصل یک‌صد و نهم تشخیص دهند، او را به رهبری انتخاب می‌کنند و در غیر این صورت یکی از آنان را به عنوان رهبر انتخاب و معرفی می‌نمایند. رهبر منتخب خبرگان، ولایت امر و همه مسئولیت‌های ناشی از آن را بر عهده خواهد داشت. رهبر در برابر قوانین با سایر افراد کشور مساوی است. 
 
\end{asl}

\begin{asl}- 
قانون مربوط به تعداد و شرایط خبرگان، کیفیت انتخاب آن‌ها و آیین‏‌نامه داخلی جلسات آنان برای نخستین دوره باید به‌وسیله فقهای اولین شورای نگهبان تهیه و با اکثریت آراء آنان تصویب شود و به تصویب نهایی رهبر انقلاب برسد. از آن پس هرگونه تغییر و تجدیدنظر در این قانون و تصویب سایر مقررات مربوط به وظایف خبرگان در صلاحیت خود آنان است.  
\end{asl}

\begin{asl}- 
شرایط و صفات رهبر 
\\
صلاحیت علمی لازم برای افتاء در ابواب مختلف فقه. 
\\
عدالت و تقوای لازم برای رهبری امت اسلام. 
\\
بینش صحیح سیاسی و اجتماعی، تدبیر، شجاعت، مدیریت و قدرت کافی برای رهبری. 
\\
در صورت تعدد واجدین شرایط فوق، شخصی که دارای بینش فقهی و سیاسی قوی‏تر باشد مقدم است. 
\end{asl}

\begin{asl}- 
وظایف و اختیارات رهبر
\\
تعیین سیاست‌ها کلی نظام جمهوری اسلامی ایران پس از مشورت با مجمع تشخیص مصلحت نظام. 
\\
نظارت بر حسن اجرای سیاست‌های کلی نظام. 
\\
فرمان همه‌‏پرسی. 
\\
فرماندهی کل نیروهای مسلح. 
\\
اعلام جنگ و صلح و بسیج نیروها.
\\
نصب و عزل و قبول استعفای‏ فقه‏های شورای نگهبان. 
\\
عالی‌ترین مقام قوه قضاییه. 
\\
رئیس سازمان صداوسیمای جمهوری اسلامی ایران. 
\\
رئیس ستاد مشترک. 
\\
فرمانده کل سپاه پاسداران انقلاب اسلامی. 
\\
فرماندهان عالی نیروهای نظامی و انتظامی. 
\\
حل اختلاف و تنظیم روابط قوای سه‌گانه. 
\\
حل معضلات نظام که از طرق عادی قابل‌حل نیست، از طریق مجمع تشخیص مصلحت نظام. 
\\
امضاء حکم ریاست جمهوری پس از انتخاب مردم صلاحیت داوطلبان ریاست جمهوری از جهت دارا بودن شرایطی که در این قانون می‌آید، باید قبل از انتخابات به تأیید شورای نگهبان و در دوره اول به تأیید رهبری برسد. 
\\
عزل رئیس‌جمهور با در نظر گرفتن مصالح کشور پس از حکم دیوان عالی کشور به تخلف وی از وظایف قانونی، یا رأی مجلس شورای اسلامی به عدم کفایت وی بر اساس اصل هشتاد و نهم. 
\\
عفو یا تخفیف مجازات محکومیت در حدود موازین اسلامی پس از پیشنهاد رئیس قوه قضاییه. 
\\
رهبر می‌تواند بعضی از وظایف و اختیارات خود را به شخص دیگری تفویض کند. 
\end{asl}

\begin{asl}- 
برکناری رهبر 
\\
هرگاه رهبر از انجام وظایف قانونی خود ناتوان شود. یا فاقد یکی از شرایط مذکور در اصول پنجم و یک‌صد و نهم گردد، یا معلوم شود از آغاز فاقد بعضی از شرایط بوده است، از مقام خود برکنار خواهد شد. تشخیص این امر به عهده خبرگان مذکور در اصل یک‌صد و هشتم است. در صورت فوت یا کناره‏گیری یا عزل رهبر، خبرگان موظف‌اند، در اسرع وقت نسبت به تعیین و معرفی رهبر جدید اقدام نمایند. تا هنگام معرفی رهبر، شورایی مرکب از رئیس‌جمهور، رئیس قوه قضاییه و یکی از فقهای شورای نگهبان انتخاب مجمع تشخیص مصلحت نظام، همه وظایف رهبری را به طور موقت به عهده می‌گیرد و چنانچه در این مدت یکی از آنان به هر دلیل نتواند انجام‌وظیفه نماید، فرد درگیر به انتخاب مجمع، با حفظ اکثریت فقهای، در شورا به جای وی منصوب می‌گردد. این شورا در خصوص وظایف بندهای ۱ و ۳ و ۵ و ۱۰ و قسمت‏های (د) و (ه) و (و) بند ۶ اصل یک‌صد و دهم، پس از تصویب سه‌چهارم اعضاء مجمع تشخیص مصلحت نظام اقدام می‌کند. هرگاه رهبر بر اثر بیماری یا حداکثر دیگری موقتاً از انجام وظایف رهبری ناتوان شود، در این مدت شورای مذکور در این اصل وظایف او را عهده‏دار خواهد بود. 

\end{asl}

\begin{asl}- 
مجمع تشخیص مصلحت نظام 
\\
مجمع تشخیص مصلحت نظام برای تشخیص مصلحت نظام برای تشخیص مصلحت در مواردی که مصوبه مجلس شورای اسلامی را شورای نگهبان خلاف موازین شرع و یا قانون اساسی بداند و مجلس با در نظر گرفتن مصلحت نظام نظر شورای نگهبان را تأمین نکند و مشاوره در اموری که رهبری به آنان ارجاع می‌دهد و سایر وظایفی که در این قانون ذکر شده است به دستور رهبری تشکیل می‌شود. اعضاء ثابت و متغیر این مجمع را مقام رهبری تعیین می‌نماید. مقررات مربوط به مجمع توسط خود اعضاء تهیه و تصویب و به تأیید مقام رهبری خواهد رسید. 
 
\end{asl}

\begin{asl}- 
پس از مقام رهبری رئیس‌جمهور عالی‌ترین مقام رسمی کشور است و مسئولیت اجرای قانون اساسی و ریاست قوه مجریه را جز در اموری که مستقیماً به رهبری مربوط می‌شود، بر عهده دارد. 
\end{asl}

\begin{asl}- 
 رئیس‌جمهور 
 \\
رئیس‌جمهور برای مدت چهار سال با رأی مستقیم مردم انتخاب می‌شود و انتخاب مجدد او به صورت متوالی تنها برای یک دوره بلامانع است. 

\end{asl}

\begin{asl}- 
رئیس‌جمهور باید از میان رجال مذهبی و سیاسی که واجد شرایط زیر باشند انتخاب گردد: 
\\
ایرانی‏الاصل، تابع ایران، مدیر و مدبر، دارای حسن سابقه و امانت و تقوی، مؤمن و معتقد به مبانی جمهوری اسلامی ایران و مذهب رسمی کشور. 
 
\end{asl}

\begin{asl}- 
نامزدهای ریاست جمهوری باید قبل از شروع انتخابات آمادگی خود را رسماً اعلام کنند. نحوه برگزاری انتخاب رئیس‌جمهوری را قانون معین می‌کند.  
\end{asl}

\begin{asl}- 
رئیس‌جمهور با اکثریت مطلق آراء شرکت‏کنندگان انتخاب می‏شود، ولی هرگاه در دوره نخست هیچ‌یک از نامزدها چنین اکثریتی به دست نیاورد، روز جمعه هفته بعد برای بار دوم رأی گرفته می‌شود. در دور دوم تنها دو نفر از نامزدها که در دور نخست‏ آراء بیشتری داشته‏اند شرکت می‌کنند، ولی اگر بعضی از نامزدهای دارنده آراء بیشتر، از شرکت در انتخابات منصرف شوند، از میان بقیه، دو نفر که در دور نخست بیش از دیگران رأی داشته‏اند برای انتخاب مجدد معرف می‌شوند.‏
\end{asl}

\begin{asl}-  
مسئولیت نظارت بر انتخابات ریاست جمهوری طبق اصل نود و نهم بر عهده شورای نگهبان است ولی قبل از تشکیل نخستین شورای نگهبان بر عهده انجمن نظارتی است که قانون تعیین می‌کند. 
\end{asl}

\begin{asl}- 
انتخاب رئیس‌جمهور جدید باید حداقل یک ماه پیش از پایان دوره ریاست جمهوری قبلی انجام شده باشد و در فاصله انتخاب رئیس‌جمهور جدید و پایان دوره ریاست جمهوری سابق، رئیس‌جمهور پیشین وظایف رئیس‌جمهوری را انجام می‌دهد. 
\end{asl}

\begin{asl}- 
هرگاه در فاصله ده روز پیش از رأی‏گیری یکی از نامزدهایی که صلاحیت او طبق این قانون احراز شده فوت کند، انتخابات به مدت دو هفته به تأخیر می‌افتد. اگر در فاصله دور نخست و دور دوم نیز یکی از دو نفر حائز اکثریت دور نخست فوت کند، انتخابات برای دو هفته تمدید می‌شود.  
\end{asl}

\begin{asl}- 
سوگند رئیس‌جمهور
\\
رئیس‌جمهور در مجلس شورای اسلامی در جلسه‏ای که با حضور رئیس قوه قضاییه و اعضای شورای نگهبان تشکیل می‏شود به ترتیب زیر سوگند یاد می‌کند و سوگندنامه را امضاء می‌نماید.
\\
بسم الله الرحمن الرحیم 
\\
«من به عنوان رئیس‌جمهور در پیشگاه قرآن کریم و در برابر ملت ایران به خداوند قادر متعال سوگند یاد می‌کنم که پاسدار مذهب رسمی و نظام جمهوری اسلامی و قانون اساسی کشور باشم و همه استعداد و صلاحیت خویش را در راه ایفای مسئولیت‌هایی که بر عهده‏ گرفته‏ام به کار گیرم و خود را وقف خدمت به مردم و اعتلای کشور، ترویج دین و اخلاق، پشتیبانی از حق و گسترش عدالت سازم و از هرگونه خودکامگی بپرهیزم و از آزادی و حرمت اشخاص و حقوقی که قانون اساسی برای ملت شناخته است حمایت کنم. در حراست از مرزها و استقلال سیاسی و اقتصادی و فرهنگی کشور از هیچ اقدامی دریغ نورزم و با استعانت از خداوند و پیروی از پیامبر اسلام و ائمه اطهار علیهم‏السلام قدرتی را که ملت به عنوان امانتی مقدس به من سپرده است همچون امینی پارسا و فداکار نگاهدار باشم و آن را به منتخب ملت پس از خود بسپارم.»

\end{asl}

\begin{asl}- 
رئیس‌جمهور در حدود اختیارات و وظایفی که به موجب قانون اساسی و یا قوانین عادی به عهده‏ دارد در برابر ملت و رهبر و مجلس شورای اسلامی مسئول است. 
\end{asl}

\begin{asl}- 
رئیس‌جمهور موظف است مصوبات مجلس یا نتیجه همه‏پرسی را پس از طی مراحل قانونی و ابلاغ به وی امضاء کند و برای اجرا در اختیار مسئولان بگذارد. 
\end{asl}

\begin{asl}- 
رئیس‌
‌تواند برای انجام وظایف قانونی خود معاونانی داشته باشد. 
معاون اول رئیس‌جمهور با موافقت وی اداره هیئت‌وزیران و مسئولیت هماهنگی سایر معاونت‌ها را به عهده خواهد داشت. 

\end{asl}

\begin{asl}- 
قراردادها و توافق‌نامه‌های بین‌المللی 
\\
امضای عهدنامه‏ها، مقاوله نامه‌ها، موافقت‌نامه‌ها و قراردادهای دولت ایران با سایر دولت‌ها و همچنین امضای پیمان‏های مربوط به اتحادیه‏های بین‏المللی پس از تصویب مجلس شورای اسلامی با رئیس‌جمهور یا نماینده قانونی او است. 

\end{asl}

\begin{asl}- 
رئیس‌جمهور مسئولیت امور برنامه و بودجه و امور اداری و استخدامی کشور را مستقیماً بر عهده دارد و می‌تواند اداره آن‌ها را به عهده دیگری بگذارد. 
\end{asl}

\begin{asl}- 
رئیس‌جمهور می‌تواند در موارد خاص، برحسب ضرورت با تصویب هیئت‌وزیران نماینده، یا نمایندگان ویژه با اختیارات مشخص تعیین نماید. در این موارد تصمیمات نماینده یا نمایندگان مذکور در حکم تصمیمات رئیس‌جمهور و هیئت‌وزیران خواهد بود. 
\end{asl}

\begin{asl}- 
سفیران به پیشنهاد وزیر امور خارجه و تصویب رئیس‌جمهور تعیین می‌شوند. رئیس‌جمهور استوارنامه سفیران را امضاء می‌کند و استوارنامه سفیران کشورهای دیگر را می‌پذیرد. 
\end{asl}

\begin{asl}- 
اعطای نشان‌های دولتی با رئیس‌جمهور است. 
\end{asl}

\begin{asl}- 
استعفای رئیس‌جمهور 
\\
رئیس‌جمهور استعفای خود را به رهبر تقدیم می‌کند و تا زمانی که استعفای او پذیرفته نشده است به انجام وظایف خود ادامه می‌دهد. 
\end{asl}

\begin{asl}- 
در صورت فوت، عزل، استعفا، غیبت یا بیماری بیش از دو ماه رئیس‌جمهور و یا در موردی که مدت ریاست جمهوری پایان یافته و رئیس‌جمهور جدید بر اثر موانعی هنوز انتخاب نشده و یا امور دیگری از این قبیل، معاون اول رئیس‌جمهور یا موافقت رهبری اختیارات و مسئولیت‌های وی را بر عهده می‌گیرد و شورایی متشکل از رئیس مجلس و رئیس قوه قضاییه و معاون اول رئیس‌جمهور موظف است ترتیبی دهد که حداکثر ظرف مدت پنجاه روز رئیس‌جمهور جدید انتخاب شود، در صورت فوت معاون اول و یا امور دیگری که مانع انجام وظایف وی گردد و نیز در صورتی که رئیس‌جمهور معاون اول نداشته باشد مقام رهبری فرد دیگری را به جای او منصوب می‌کند. 
\end{asl}

\begin{asl}- 
در مدتی که اختیارات و مسئولیت‌های رئیس‌جمهور بر عهده معاون اول یا فرد دیگری است که به موجب اصل یک‌صد و سی و یکم منصوب می‌گردد، وزرا را نمی‌توان استیضاح کرد یا به آنان رأی عدم اعتماد داد و نیز نمی‌توان برای تجدیدنظر در قانون اساسی و یا امر همه‏پرسی اقدام نمود. 
\end{asl}

\begin{asl}- 
وزرا توسط رئیس‌جمهور تعیین و برای رفتن رأی اعتماد به مجلس معرفی می‌شوند با تغییر مجلس، گرفتن رأی اعتماد جدید برای وزرا لازم نیست. تعداد وزیران و حدود اختیارات هر یک از آنان را قانون معین می‌کند. 
\end{asl}

\begin{asl}- 
ریاست هیئت‌وزیران با رئیس‌جمهور است که بر کار وزیران نظارت دارد و با اتخاذ تدابیر لازم به هماهنگ ساختن تصمیم‏های وزیران و هیئت دولت می‌پردازد و با همکاری وزیران، برنامه و خط‏مشی دولت را تعیین و قوانین را اجرا می‌کند. در موارد اختلاف‌نظر و یا تداخل در وظایف قانونی دستگا‏ه‏های دولتی در صورتی که نیاز به تفسیر یا تغییر قانون نداشته باشد، تصمیم هیئت‌وزیران که به پیشنهاد رئیس‌جمهور اتخاذ می‌شود لازم‏الاجرا است. رئیس‌جمهور در برابر مجلس مسئول اقدامات هیئت‌وزیران است.  
\end{asl}

\begin{asl}- 
وزرا تا زمانی که عزل نشده‏اند و یا بر اثر استیضاح یا درخواست رأی اعتماد، مجلس به آن‌ها رأی عدم اعتماد نداده است در سمت خود باقی می‌مانند. استعفای هیئت‌وزیران یا هر یک از آنان به رئیس‌جمهور تسلیم می‌شود و هیئت‌وزیران تا تعیین دولت جدید به وظایف خود ادامه خواهند داد. رئیس‌جمهور می‌تواند برای وزارتخانه‏هایی که وزیر ندارند حداکثر برای مدت سه ماه سرپرست تعیین نماید. 
\end{asl}

\begin{asl}- 
رئیس‌جمهور می‌تواند وزرا را عزل کند و در این صورت باید برای وزیر یا وزیران جدید از مجلس رأی اعتماد بگیرد و در صورتی که پس از ابراز اعتماد مجلس به دولت نیمی از هیئت‌وزیران تغییر نماید باید مجدداً از مجلس شورای اسلامی برای هیئت‌وزیران تقاضای رأی اعتماد کند. 
\end{asl}

\begin{asl}- 
هر یک از وزیران مسئول وظایف خاص خویش در باربر رئیس‌جمهور و مجلس است و در اموری که به تصویب هیئت‌وزیران می‌رسد مسئول اعمال دیگران نیز هست.  
\end{asl}

\begin{asl}- 
علاوه بر مواردی که هیئت‌وزیران یا وزیری مأمور تدوین آیین‏نامه‏های اجرایی قوانین می‌شود، هیئت‌وزیران حق دارد برای انجام وظایف اداری و تأمین اجرای قوانین و تنظیم سازمان‌های اداری به وضع تصویب‏نامه و آیین‏نامه بپردازد. هر یک از وزیران نیز در حدود وظایف خویش و مصوبات هیئت‌وزیران حق وضع آیین‏نامه و صدور بخشنامه را دارد ولی مفاد این مقررات نباید با متن و روح قوانین مخالف باشد. 
دولت می‌تواند تصویب برخی از امور مربوط به وظایف خود را به کمیسیون‌های متشکل از چند وزیر واگذار نماید. مصوبات این کمیسیون‌ها در محدوده قوانین پس از تأیید رئیس‌جمهور لازم‏الاجرا است. 
تصویب‏نامه‏ها و آیین‏نامه‏های دولت و مصوبات کمیسیون‌های مذکور در این اصل‏، ضمن ابلاغ برای اجرا به اطلاع رئیس مجلس شورای اسلامی می‌رسد تا در صورتی که آن‌ها را بر خلاف قوانین بیابد با ذکر دلیل برای تجدیدنظر به هیئت‌وزیران بفرستند. 
\end{asl}

\begin{asl}- 
اصلح دعاوی راجع به اموال عمومی و دولتی یا ارجاع آن به داوری در هر مورد، موکول به تصویب هیئت‌وزیران است و باید به اطلاع مجلس برسد. در مواردی که طرف دعوی خارجی باشد و در موارد مهم داخلی باید به تصویب مجلس نیز برسد. موارد مهم را قانون تعیین می‌کند. 
\end{asl}

\begin{asl}- 
رسیدگی به اتهام رئیس‌جمهور و معاونان او و وزیران در مورد جرائم عادی با اطلاع مجلس شورای اسلامی در دادگاه‌های عمومی دادگستری انجام می‌شود. 
\end{asl}

\begin{asl}- 
رئیس‌جمهور، معاونان رئیس‌جمهور، وزیران و کارمندان دولت نمی‌توانند بیش از یک شغل دولتی داشته باشند و داشتن هر نوع شغل دیگر در مؤسساتی که تمام یا قسمتی از سرمایه آن متعلق به دولت یا مؤسسات عمومی است و نمایندگی مجلس شورای اسلامی و وکالت دادگستری و مشاوره حقوقی و نیز ریاست و مدیریت عامل یا عضویت در هیئت‌مدیره انواع مختلف شرکت‌های خصوصی، جز شرکت‌های تعاونی ادارات و مؤسسات برای آنان ممنوع است. سمت‌های آموزشی در دانشگاه‏ها و مؤسسات تحقیقاتی از این، حکم مستثنی است. 
\end{asl}

\begin{asl}- 
دارایی رهبر، رئیس‌جمهور، معاونان رئیس‌جمهور، وزیران و همسر و فرزندان آنان قبل و بعد از خدمت، توسط رئیس قوه قضاییه رسیدگی می‌شود که بر خلاف حق، افزایش نیافته باشد. 
\end{asl}

\begin{asl}- 
ارتش جمهوری اسلامی ایران پاسداری از استقلال و تمامیت ارضی و نظام جمهوری اسلامی کشور را بر عهده دارد. 
\end{asl}

\begin{asl}- 
ارتش جمهوری اسلامی ایران باید ارتشی اسلامی باشد که ارتشی مکتبی و مردمی است و باید افرادی شایسته را به خدمت بپذیر که به اهداف انقلاب اسلامی مؤمن و در راه تحقق آن فداکار باشند. 
\end{asl}

\begin{asl}- 
هیچ فرد خارجی به عضویت در ارتش و نیروهای انتظامی کشور پذیرفته نمی‌شود. 
\end{asl}

\begin{asl}- 
استقرار هرگونه پایگاه نظامی خارجی در کشور هرچند به عنوان استفاده‏های صلح‏آمیز باشد ممنوع است.
\end{asl}

\begin{asl}- 
دولت باید در زمان صلح از افراد و تجهیزات فنی ارتش در کارهای امدادی، آموزشی، تولیدی و جهاد سازندگی، با رعایت کامل موازین عدل اسلامی استفاده کند در حدی که به آمادگی رزمی آسیبی وارد نیاید. 
\end{asl}

\begin{asl}- 
هر نوع بهره‏برداری شخصی از وسایل و امکانات ارتش و استفاده شخصی از افراد آن‌ها به صورت گماشته، راننده شخصی و نظایر این‌ها ممنوع است. 
\end{asl}

\begin{asl}- 
ترفیع درجه نظامیان و سلب آن به موجب قانون است. 
\end{asl}

\begin{asl}- 
سپاه پاسداران انقلاب اسلامی که در نخستین روزهای پیروزی این انقلاب تشکیل شد، برای ادامه نقش خود در نگهبانی از انقلاب و دستاوردهای آن پابرجا می‌ماند. حدود وظایف و قلمرو مسئولیت این سپاه در رابطه با وظایف و قلمرو مسئولیت نیروهای مسلح دیگر با تأکید بر همکاری و هماهنگی برادرانه میان آن‌ها به‌وسیله قانون تعیین می‌شود.  
\end{asl}

\begin{asl}- 
به حکم آیه کریمیه "واعدوالهم مااستطعتم من قوه و من رباط‏الخیل ترهبون به عدوالله و عدوکم و آخرین من دونهم لاتعلمونهم‏الله یعلمهم" دولت موظف است برای همه افراد کشور برنامه و امکانات آموزش نظامی را بر طبق موازین اسلامی فراهم نماید، به طوری که همه افراد همواره توانایی دفاع مسلحانه از کشور و نظام جمهوری اسلامی ایران را داشته باشند، ولی داشتن اسلحه باید با اجازه مقامات رسمی باشد. 
\end{asl}

\begin{asl}- 
سیاست خارجی جمهوری اسلامی ایران بر اساس نفی هرگونه سلطه‏جویی و سلطه‏پذیری، حفظ استقلال همه‌جانبه و تمامیت ارضی کشور، دفاع از حقوق همه مسلمانان و عدم تعهد در برابر قدرت‏های سلطه‏گر و روابط صلح‏آمیز متقابل با دول غیر محارب استوار است.  
\end{asl}

\begin{asl}- 
هرگونه قرارداد که موجب سلطه بیگانه بر منابع طبیعی و اقتصادی، فرهنگ، ارتش و دیگر شیون کشور گردد ممنوع است.  
\end{asl}

\begin{asl}- 
جمهوری اسلامی ایران سعادت انسان در کل جامعه بشری را آرمان خود می‌داند و استقلال و آزادی و حکومت حق و عدل را حق همه مردم جهان می‌شناسد؛ بنابراین در عین خودداری کامل از هرگونه دخالت در امور داخلی ملت‌های دیگر، از مبارزه حق‏طلبانه مستضعفین در برابر مستکبرین در هر نقطه از جهان حمایت می‌کند.  
\end{asl}

\begin{asl}- 
دولت جمهوری اسلامی ایران می‌تواند به کسانی که پناهندگی سیاسی بخواهند پناه دهد مگر این که بر طبق قوانین ایران خائن و تبهکار شناخته شوند. 
\end{asl}

\begin{asl}- 
وظایف قوه قضاییه 
\\
قوه قضاییه قوه‌ای است مستقل که پشتیبان حقوق فردی و اجتماعی و مسئول تحقق بخشند به عدالت و عهده‌دار وظایف زیر است‏: 
\begin{enumerate}
	\item 
	رسیدگی و صدور حکم در مورد تظلمات، تعدیات، شکایات، حل‌وفصل دعاوی و رفع خصومات و اخذ تصمیم و اقدام لازم در آن قسمت از امور حسبیه که قانون معین می‌کند. 
	\item
	احیای حقوق عامه و گسترش عدل و آزادی‌های مشروع.
	\item
	نظارت بر حسن اجرای قوانین.
	\item
	کشف جرم و تعقیب مجازات و تعزیر مجرمین و اجرای حدود و مقررات مدون جزایی اسلام.
	\item
	اقدام مناسب برای پیشگیری از وقوع جرم و اصلاح مجرمین. 
\end{enumerate}
\end{asl}

\begin{asl}- 
به منظور انجام مسئولیت‏های قوه قضاییه در کلیه امور قضایی و اداری و اجرایی قمام رهبری یک نفر مجتهد عادل و آگاه به امور قضایی و مدیر و مدبر را برای مدت پنج سال به عنوان رئیس قوه قضاییه تعیین می‌نماید که عالی‌ترین مقام قوه قضاییه است. 
\end{asl}

\begin{asl}- 
وظایف رئیس قوه قضاییه به شرح زیر است: 
\\
ایجاد تشکیلات لازم در دادگستری به تناسب مسئولیت‏های اصل یک‌صد و پنجاه و ششم. 
\\
تهیه لوایح قضایی متناسب با جمهوری اسلامی. 
\\
استخدام قضات عادل و شایسته و عزل و نصب آن‌ها و تغییر محل مأموریت و تعیین مشاغل و ترفیع آنان و مانند این‌ها از امور اداری، طبق قانون. 
 
\end{asl}

\begin{asl}- 
مرجع رسمی تظلمات و شکایات، دادگستری است. تشکیل دادگاه‌ها و تعیین صلاحیت آن‌ها منوط به حکم قانون است.  
\end{asl}

\begin{asl}- 
وزیر دادگستری مسئولیت کلیه مسائل مربوطه به روابط قوه قضاییه با قوه مجریه و قوه مقننه را بر عهده دارد و از میان کسانی که رئیس قوه قضاییه به رئیس‌جمهور پیشنهاد می‌کند، انتخاب می‌گردد. 
رئیس قوه قضاییه می‌تواند اختیارات تام مالی و اداری و نیز اختیارات استخدامی غیر قضات را به وزیر دادگستری تفویض کند. در این صورت وزیر دادگستری دارای همان اختیارات و وظایفی خواهد بود که در قوانین برای وزرا به عنوان عالی‌ترین مقام اجرایی پیش‏بینی می‌شود. 
\end{asl}

\begin{asl}- 
دیوان عالی کشور به منظور نظارت بر اجرای صحیح قوانین در محاکم و ایجاد وحدت رویه قضایی و انجام مسئولیت‌هایی که طبق قانون به آن محول می‌شود بر اساس ضوابطی که رئیس قوه قضاییه تعیین می‌کند تشکیل می‌گردد. 
\end{asl}

\begin{asl}- 
رئیس دیوان عالی کشور و دادستان کل باید مجتهد عادل و آگاه به امور قضایی باشند و رئیس قوه قضاییه با مشورت قضایت دیوان عالی کشور آن‌ها را برای مدت پنج سال به این سمت منصوب می‌کند. 
\end{asl}

\begin{asl}- 
صفات و شرایط قاضی طبق موازین فقهی به‌وسیله قانون معین می‌شود.  
\end{asl}

\begin{asl}- 
قاضی را نمی‌توان از مقامی که شاغل آن است بدون محاکمه و ثبوت جرم یا تخلفی که موجب انفصال است به طور موقت یا دائم منفصل کرد یا بدون رضای او محل خدمت یا سمتش را تغییر داد مگر به اقتضای مصلحت جامعه با تصمیم رئیس قوه قضاییه پس از مشورت با رئیس دیوان عالی کشور و دادستان کل. نقل‌وانتقال دوره‌ای قضات بر طبق ضوابط کلی که قانون تعیین می‌کند، صورت می‌گیرد. 
\end{asl}

\begin{asl}- 
محاکمات، علنی انجام می‌شود و حضور افراد بلامانع است مگر آن که به تشخیص دادگاه، علنی بودن آن منافی عفت عمومی یا نظم عمومی باشد یا در دعاوی خصوصی طرفین دعوا تقاضا کنند که محاکمه علنی نباشد. 
\end{asl}

\begin{asl}- 
احکام دادگاه‌ها باید مستدل و مستند به مواد قانون و اصولی باشد که بر اساس آن حکم صادر شده است. 
\end{asl}

\begin{asl}- 
قاضی موظف است کوشش کند حکم هر دعوا را در قوانین مدونه بیابد و اگر نیابد با استناد به منابع معتبر اسلامی یا فتاوای معتبر، حکم قضیه را صادر نماید و نمی‌تواند به بهانه سکوت یا نقص یا اجمال یا تعارض قوانین مدونه از رسیدگی به دعوا و صدور حکم امتناع ورزد. 
\end{asl}

\begin{asl}- 
رسیدگی به جرائم سیاسی و مطبوعاتی علنی است و با حضور هیئت‌منصفه در محاکم دادگستری صورت می‌گیرد. نحوه انتخاب، شرایط، اختیارات هیئت‌منصفه و تعریف جرم سیاسی را قانون ر اساس موازین اسلامی معین می‌کند. 
\end{asl}

\begin{asl}- 
هیچ فعلی یا ترک فعلی به استناد قانونی که بعد از آن وضع شده است جرم محسوب نمی‌شود. 
\end{asl}

\begin{asl}- 
قضات دادگاه‌ها مکلف‌اند از اجرای تصویب‏‌نامه‌ها و آیین‌نامه‌های دولتی که مخالف با قوانین و مقررات اسلامی یا خارج از حدود اختیارات قوه مجریه است خودداری کنند و هر کس می‌تواند ابطال این‌گونه مقررات را از دیوان عدالت اداری تقاضا کند.  
\end{asl}

\begin{asl}- 
هرگاه در اثر تفسیر یا اشتباه قاضی در موضوع یا در حکم یا در تطبیق حکم بر مورد خاص، ضرر مادی یا معنوی متوجه کسی گردد، در صورت تقصر، مقصر طبق موازین اسلامی ضامن است و ر غیر این صورت خسارت به‌وسیله دولت جبران می‌شود و در هر حال از متهم اعاده حیثیت می‌گردد. 
\end{asl}

\begin{asl}- 
برای رسیدگی به جرائم مربوط به وظایف خاص نظامی یا انتظامی اعضاء ارتش، ژاندارمری، شهربانی و سپاه پاسداران انقلاب اسلامی، محاکم نظامی مطابق قانون تشکیل می‌گردد، ولی به جرائم عمومی آنان یا جرائمی که در مقام ضابط دادگستری مرتکب شوند در محاکم عمومی رسیدگی می‌شود. دادستانی و دادگاه‌های نظامی، بخشی از قوه قضاییه کشور و مشمول اصول مربوط به این قوه هستند.  
\end{asl}

\begin{asl}- 
به منظور رسیدگی به شکایات، تظلمات و اعتراضات مردم نسبت به مأمورین یا واحدها با آیین‌نامه‌های دولتی و احقاق حقوق آن‌ها، دیوانی به نام "دیوان عدالت اداری " زیر نظر رئیس قوه قضاییه تأسیس می‌گردد. حدود اختیارات و نحوه عمل این دیوان را قانون تعیین می‌کند.  
\end{asl}

\begin{asl}- 
بر اساس حق نظارت قوه قضاییه نسبت به حسن جریان امور و اجرای صحیح قوانین در دستگاه‌های اداری سازمانی به نام "سازمان بازرسی کل کشور " زیر نظر رئیس قوه قضاییه تشکیل می‌گردد. حدود اختیارات و وظایف این سازمان را قانون تعیین می‌کند. 
\end{asl}

\begin{asl}- 
در صداوسیمای جمهوری اسلامی ایران، آزادی بیان و نشر افکار با رعایت موازین اسلامی و مصالح کشور باید تأمین گردد. نصب و عزل رئیس سازمان صداوسیمای جمهوری اسلامی ایران با مقام رهبری است و شورایی مرکب از نمایندگان رئیس‌جمهور و رئیس قوه قضاییه و مجلس شورای اسلامی (هرکدام دو نفر) نظارت بر این سازمان خواهند داشت. خط‌مشی و ترتیب اداره سازمان و نظارت بر آن را قانون معین می‌کند.  
\end{asl}

\begin{asl}- 
وظایف شورای عالی امنیت ملی 
\\
به منظور تأمین منافع ملی و پاسداری از انقلاب اسلامی و تمامیت ارضی و حاکمیت ملی "شورای عالی امنیت ملی " به ریاست رئیس‌جمهور، با وظایف زیر تشکیل می‌گردد:
\renewcommand{\labelitemi}{$-$}
\begin{itemize}
	\item 
	تعیین سیاست‌های دفاعی - امنیتی کشور در محدوده سیاست‌های کلی تعیین شده از طرف مقام رهبری. 
	\item
	هماهنگ نمودن فعالیت‏های سیاسی، اطلاعاتی، اجتماعی، فرهنگی و اقتصادی در ارتباط با تدابیر کلی دفاعی - امنیتی. 
	\item
		بهره‏گیری از امکانات مادی و معنوی کشور برای مقابله با تهدیدهای داخلی و خارجی.
\end{itemize}
اعضای شورا عبارت‌اند از:
\begin{enumerate}
	\item 
	روسای قوای سه‌گانه 
	\item
	رئیس ستاد فرماندهی کل نیروهای مسلح 	
	\item
	مسئول امور برنامه و بودجه 
	\item
	دو نماینده به انتخاب مقام رهبری 
	\item
	وزرای امور خارجه، کشور، اطلاعات 
	\item
	حسب مورد وزیر مربوط و عالی‌ترین مقام ارتش و سپاه 
	
\end{enumerate} 

شورای عالی امنیت ملی به تناسب وظایف خود شوراهای فرعی از قبیل شورای دفاع و شورای امنیت کشور تشکیل می‌دهد. ریاست هر یک از شوراهای فرعی با رئیس‌جمهور یا یکی از اعضای شورای عالی است که از طرف رئیس‌جمهور تعیین می‌شود. حدود اختیارات و وظایف شوراهای فرعی را قانونی معین می‌کند و تشکیلات آن‌ها به تصویب شورای عالی می‌رسد. مصوبات شورای عالی امنیت ملی پس از تأیید مقام رهبری قابل‌اجراست.

\end{asl}

\begin{asl}- 
بازنگری در قانون اساسی
\\
بازنگری در قانون اساسی جمهوری اسلامی ایران، در موارد ضروری به ترتیب زیر انجام می‌گیرد. مقام رهبری پس از مشورت با مجمع تشخیص مصلحت نظام طی حکمی خطاب به رئیس‌جمهور موارد اصلاح یا تتمیم قانون اساسی را به شورای بازنگری قانون اساسی با ترکیب زیر پیشنهاد می‌نماید: 
\renewcommand{\labelitemi}{$-$}
\begin{itemize}
	\item 
اعضای شورای نگهبان. 
	\item 
روسای قوای سه‌گانه. 
	\item 
اعضای ثابت مجمع تشخیص مصلحت نظام. 
	\item 
پنج نفر از اعضای مجلس خبرگان رهبری. 
	\item 
ده نفر به انتخاب مقام رهبری. 
	\item 
سه نفر از هیئت‌وزیران. 
	\item 
سه نفر از قوه قضاییه
\end{itemize}
\end{asl}


	
	
	
	
\end{document}
